\documentclass{article}
\usepackage[utf8]{inputenc}
\usepackage[T1]{fontenc}
\usepackage{geometry}
\geometry{a4paper, margin=2.5cm}
\usepackage{amsmath}
\usepackage{amssymb}
\usepackage{listings}
\usepackage{hyperref}
\usepackage{graphicx}
\usepackage{booktabs}
\usepackage{array}

\title{Wochenbericht XX - Praktikum "Verteilte Systeme": Titel}
\author{Jannik Schön}
\date{\today}

\lstset{
    language=Java,                
    basicstyle=\footnotesize\ttfamily,
    numbers=left,                  
    numberstyle=\tiny\color{gray},
    stepnumber=1,                   
    numbersep=5pt,                 
    backgroundcolor=\color{white}, 
    showspaces=false,               
    showstringspaces=false,         
    showtabs=false,                
    frame=single,                 
    tabsize=4,                      
    captionpos=b,                   
    breaklines=true,                
    breakatwhitespace=true,        
    title=\lstname,               
    escapeinside={\%*}{*)},         
    morekeywords={Integer,IntWritable, Iterable, Text},
}

\begin{document}
\maketitle
\section{Mitglieder des Projektes }

\begin{tabular}{>{\raggedright\arraybackslash}p{3cm} >{\raggedright\arraybackslash}p{4cm} >{\centering\arraybackslash}p{2cm} >{\centering\arraybackslash}p{2cm} >{\raggedright\arraybackslash}p{3cm}}
\toprule
\textbf{Mitglied des Projektes} & \textbf{Aufgabe} & \textbf{Fortschritt} & \textbf{Zeiteinsatz} & \textbf{Check} \\
\midrule
Manh-An David Dao & Testen & 75\% & 2h & in progress \\
\hline
Jannik Schön & Testen & 75\% & 2h & in progress \\
\hline
Marc Siekmann & Testen & 75\% & 9h & in progress \\
\hline
Philipp Patt &  Testen & 75\% & 9
h & in progress\\

\bottomrule
\end{tabular}

\section{Zusammenfassung der Woche}

In dieser Woche wurden Tests spezifiziert und anschließend durchgeführt, um deren Funktionalität und Zuverlässigkeit zu überprüfen. 
Im Zuge dieser Testaktivitäten wurden zudem verschiedene Bugfixes vorgenommen, um auftretende Fehler zu beheben. 
Darüber hinaus erfolgte eine Überarbeitung beider arc42-Dokumente.
 
\section{Nächste Schritte}
- Testen \\ 
- letzte Anpassungen im Arc42

\end{document}
