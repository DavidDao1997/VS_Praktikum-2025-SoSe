\documentclass{article}
\usepackage[utf8]{inputenc}
\usepackage[T1]{fontenc}
\usepackage{geometry}
\geometry{a4paper, margin=2.5cm}
\usepackage{amsmath}
\usepackage{amssymb}
\usepackage{listings}
\usepackage{hyperref}
\usepackage{graphicx}
\usepackage{booktabs}
\usepackage{array}

\title{Wochenbericht V - Praktikum "Verteilte Systeme": Trennung Applikation und Middleware}
\author{Name Protokollführer}
\date{\today}

\lstset{
    language=Java,                
    basicstyle=\footnotesize\ttfamily,
    numbers=left,                  
    numberstyle=\tiny\color{gray},
    stepnumber=1,                   
    numbersep=5pt,                 
    backgroundcolor=\color{white}, 
    showspaces=false,               
    showstringspaces=false,         
    showtabs=false,                
    frame=single,                 
    tabsize=4,                      
    captionpos=b,                   
    breaklines=true,                
    breakatwhitespace=true,        
    title=\lstname,               
    escapeinside={\%*}{*)},         
    morekeywords={Integer,IntWritable, Iterable, Text},
}

\begin{document}
\maketitle
\section{Mitglieder des Projektes }

\begin{tabular}{>{\raggedright\arraybackslash}p{3cm} >{\raggedright\arraybackslash}p{4cm} >{\centering\arraybackslash}p{2cm} >{\centering\arraybackslash}p{2cm} >{\raggedright\arraybackslash}p{3cm}}
\toprule
\textbf{Mitglied des Projektes} & \textbf{Aufgabe} & \textbf{Fortschritt} & \textbf{Zeiteinsatz} & \textbf{Check} \\
\midrule
Manh-An David Dao & Überarbeitung arc42 chapter 6 & 80\% & 3h & <in review, in progress> \\
\hline
Jannik Schön &  & 0\% & 0h & <ok, in review, in progress> \\
\hline
Marc Siekmann &  Überarbeitung arc42 Chapter 1 - 4 Middleware  & 80\% & 4h & in review \\
\hline
Phillip Patt &  & 0\% & 0h & <ok, in review, in progress>\\

\bottomrule
\end{tabular}

\section{Zusammenfassung der Woche}

In dieser Woche fand die Bearbeitung 
\\\\
Wesentliche Pull request/commits des Projektes waren: \\ \\


\section{Bearbeitete Themen und Schlüssel Erkenntnisse}

\subsection{Überarbeitung arc42 chapter 6}
In den vorherigen Versionen der Sequenzdiagramme wurden zahlreiche Unstimmigkeiten festgestellt.
In den Kapiteln 4 und 5 zeigte sich, dass einige Funktionen noch unklar definiert waren und die Zuordnung der vorhandenen Komponenten sowie ihrer Schnittstellen nicht eindeutig war.
Die Sequenzdiagramme wurden daraufhin überarbeitet und an die definierten Funktionen und Komponenten angepasst.
\\
Ziel dieser Anpassungen war es, eine höhere Konsistenz zwischen den verschiedenen Sichten herzustellen und das Verständnis der Gesamtarchitektur zu verbessern.

\subsection{Aufgabe}


\subsection{Aufgabe Überarbeitung arc42 Chapter 1 - 4 Middleware}
Vorherige Version war zu stark mit Applikation vermischt. Eine komplette Trennung der Middleware als austauschbare Schicht ist erforderlich (TODO QUELLE). Die Kapitel 1 - 4 wurden daher überarbeitet. Die Middleware ist nun als reine Vermittlungsschicht aufgebaut, die sich um die Verteilungstransparenzen und das Marshalling kümmert. Über den Watchdog muss einmal Rücksprache gehalten werden, da dieser unabhängig von Middleware und Applikation existiert (TODO QUELLE) 

\subsection{Aufgabe}


\section{Bezug zum Skript}

\begin{itemize}
	\item \textbf{Skript Abschnitt}
		\begin{itemize}
			\item 
		\end{itemize}
	\item \textbf{Skript Abschnitt}
		\begin{itemize}
			\item 
		\end{itemize}
\end{itemize}
 
\section{Nächste Schritte}
Middleware:\\
- Blocksicht und Architekturentscheidungen (Proxy und Name Service)\\
- Verlaufssicht \\
- Watchdog \\





\end{document}
