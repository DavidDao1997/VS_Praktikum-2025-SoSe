\documentclass{article}
\usepackage[utf8]{inputenc}
\usepackage[T1]{fontenc}
\usepackage{geometry}
\geometry{a4paper, margin=2.5cm}
\usepackage{amsmath}
\usepackage{amssymb}
\usepackage{listings}
\usepackage{hyperref}
\usepackage{graphicx}
\usepackage{booktabs}
\usepackage{array}

\title{Wochenbericht V - Praktikum "Verteilte Systeme": Trennung Applikation und Middleware}
\author{Marc Siekmann}
\date{\today}

\lstset{
    language=Java,                
    basicstyle=\footnotesize\ttfamily,
    numbers=left,                  
    numberstyle=\tiny\color{gray},
    stepnumber=1,                   
    numbersep=5pt,                 
    backgroundcolor=\color{white}, 
    showspaces=false,               
    showstringspaces=false,         
    showtabs=false,                
    frame=single,                 
    tabsize=4,                      
    captionpos=b,                   
    breaklines=true,                
    breakatwhitespace=true,        
    title=\lstname,               
    escapeinside={\%*}{*)},         
    morekeywords={Integer,IntWritable, Iterable, Text},
}

\begin{document}
\maketitle
\section{Mitglieder des Projektes }

\begin{tabular}{>{\raggedright\arraybackslash}p{3cm} >{\raggedright\arraybackslash}p{4cm} >{\centering\arraybackslash}p{2cm} >{\centering\arraybackslash}p{2cm} >{\raggedright\arraybackslash}p{3cm}}
\toprule
\textbf{Mitglied des Projektes} & \textbf{Aufgabe} & \textbf{Fortschritt} & \textbf{Zeiteinsatz} & \textbf{Check} \\
\midrule
Manh-An David Dao & Überarbeitung arc42 Chapter 6 & 100\%& 3h & ok \\
\hline
Jannik Schön & Überarbeitung arc42 Chapter 5 & 80\% & 3h & in review \\
\hline
Marc Siekmann &  Überarbeitung arc42 Chapter 1 - 4 Middleware  & 80\% & 4h & in review \\
\hline
Philipp Patt & Überarbeitung arc42 Chapter 4 & 80\% & 4h & in review\\

\bottomrule
\end{tabular}

\section{Zusammenfassung der Woche}

In dieser Woche fand die inhaltliche Trennung der Applikation und der Middleware statt. Demnach wurden beide arc42 Dokumentationen überarbeitet.
\\\\
Wesentliche pull-requests/commits sind in den Fußnoten hinterlegt.  \\ \\


\section{Bearbeitete Themen und Schlüssel Erkenntnisse}

\subsection{Überarbeitung arc42 Chapter 4-6}
In der vorherigen Woche wurden noch einige Unstimmigkeiten erkannt. 
Dabei waren Funktionssignaturen, Bausteine, Schnittstellen der Bausteine und die daraus resultierenden Sequenzdiagramme noch nicht kohärent.
Diese Woche wurde durch gemeinsame Diskussion diese Unstimmigkeiten behoben, so dass die Kapitel nun aufeiander aufbauend sind. 
Außerdem wird die Middleware zunächst vollständig vernachlässigt. Sobald die Verteilungsschicht thematisiert wird, spielt sie wieder eine Rolle.
\\\\
Die Lösungsstrategie ist nun in Tabellenform und grenzt die Schnittstellen sauberer voneinander ab.\footnote{\url{https://git.haw-hamburg.de/infwgi246/vs_praktikum-2025-sose/-/merge_requests/9}}
\\\\
Die Bausteine orientieren sich an einem MVC-Pattern \footnote{\url{https://github.com/scimbe/vs_script/blob/main/vs-script-first-v01.pdf} S. 91 ff. } und es wurden klaren Schnittstellen zwischen den Bausteinen definiert \footnote{\url{https://git.haw-hamburg.de/infwgi246/vs_praktikum-2025-sose/-/merge_requests/17}}.
\\\\
Die Sequenzdiagramme wurden anhand der Lösungsstrategie und der Bausteinsicht angepasst.\footnote{\url{https://git.haw-hamburg.de/infwgi246/vs_praktikum-2025-sose/-/merge_requests/13}}\\\\
Ziel dieser Anpassungen war es, eine höhere Konsistenz zwischen den verschiedenen Sichten herzustellen und das Verständnis der Gesamtarchitektur zu verbessern.




\subsection{Aufgabe Überarbeitung arc42 Chapter 1 - 4 Middleware}
Vorherige Version war zu stark mit Applikation vermischt. Eine komplette Trennung der Middleware als austauschbare Schicht ist erforderlich \footnote{\url{https://github.com/scimbe/vs_script/blob/main/vs-script-first-v01.pdf} S. 80 }. 
Die Kapitel 1 - 4 wurden daher überarbeitet. 
Die Middleware ist nun als reine Vermittlungsschicht aufgebaut, die sich um die Verteilungstransparenzen und das Marshalling \footnote{\url{https://github.com/scimbe/vs_script/blob/main/vs-script-first-v01.pdf} S. 123 ff. } kümmert. 
Über den Watchdog muss einmal Rücksprache gehalten werden \footnote{\url{https://github.com/scimbe/vs_script/blob/main/vs-script-first-v01.pdf} S. 115 ff. }. Diese Änderungen wurden eingeflegt.
\footnote{\url{https://git.haw-hamburg.de/infwgi246/vs_praktikum-2025-sose/-/merge_requests/11}}






 
\section{Nächste Schritte}
Middleware:\\
- Bausteinsicht mund Architekturentscheidungen (Proxy\footnote{\url{https://github.com/scimbe/vs_script/blob/main/vs-script-first-v01.pdf} S. 95 ff. } und Service Discovery \footnote{\url{https://github.com/scimbe/vs_script/blob/main/vs-script-first-v01.pdf} S. 212 ff. }, Client-Server \footnote{\url{https://github.com/scimbe/vs_script/blob/main/vs-script-first-v01.pdf} S. 118 ff. } )\\
- Laufzeitsicht \\
- Watchdog \\
\\
Applikation:\\
- Modell Businesslogik ausarbeiten\\
- Verteilungsschicht \\
\\
Generell:\\
MVP für RPC-Aufrufe erstellen




\end{document}
