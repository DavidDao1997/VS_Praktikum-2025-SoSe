\documentclass{article}
\usepackage[utf8]{inputenc}
\usepackage[T1]{fontenc}
\usepackage{geometry}
\geometry{a4paper, margin=2.5cm}
\usepackage{amsmath}
\usepackage{amssymb}
\usepackage{listings}
\usepackage{hyperref}
\usepackage{graphicx}
\usepackage{booktabs}
\usepackage{array}

\title{Wochenbericht XX - Praktikum "Verteilte Systeme": Titel}
\author{Name Protokollführer}
\date{\today}

\lstset{
    language=Java,                
    basicstyle=\footnotesize\ttfamily,
    numbers=left,                  
    numberstyle=\tiny\color{gray},
    stepnumber=1,                   
    numbersep=5pt,                 
    backgroundcolor=\color{white}, 
    showspaces=false,               
    showstringspaces=false,         
    showtabs=false,                
    frame=single,                 
    tabsize=4,                      
    captionpos=b,                   
    breaklines=true,                
    breakatwhitespace=true,        
    title=\lstname,               
    escapeinside={\%*}{*)},         
    morekeywords={Integer,IntWritable, Iterable, Text},
}

\begin{document}
\maketitle
\section{Mitglieder des Projektes }

\begin{tabular}{>{\raggedright\arraybackslash}p{3cm} >{\raggedright\arraybackslash}p{4cm} >{\centering\arraybackslash}p{2cm} >{\centering\arraybackslash}p{2cm} >{\raggedright\arraybackslash}p{3cm}}
\toprule
\textbf{Mitglied des Projektes} & \textbf{Aufgabe} & \textbf{Fortschritt} & \textbf{Zeiteinsatz} & \textbf{Check} \\
\midrule
Manh-An David Dao &  & 0\% & 0h & <ok, in review, in progress> \\
\hline
Jannik Schön &  & 0\% & 0h & <ok, in review, in progress> \\
\hline
Marc Siekmann &  & 0\% & 0h & <ok, in review, in progress> \\
\hline
Philipp Patt &  & 0\% & 0h & <ok, in review, in progress>\\

\bottomrule
\end{tabular}

\section{Zusammenfassung der Woche}

In dieser Woche fand die Bearbeitung 
\\\\
Wesentliche pull-requests/commits sind in den Fußnoten hinterlegt. \\ \\


\section{Bearbeitete Themen und Schlüssel Erkenntnisse}

\subsection{Aufgabe}


\subsection{Aufgabe}


\subsection{Aufgabe}


\subsection{Aufgabe}


\section{Bezug zum Skript}

\begin{itemize}
	\item \textbf{Skript Abschnitt}
		\begin{itemize}
			\item 
		\end{itemize}
	\item \textbf{Skript Abschnitt}
		\begin{itemize}
			\item 
		\end{itemize}
\end{itemize}
 
\section{Nächste Schritte}

\end{document}
