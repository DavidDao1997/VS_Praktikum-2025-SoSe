\documentclass{article}
\usepackage[utf8]{inputenc}
\usepackage[T1]{fontenc}
\usepackage{geometry}
\geometry{a4paper, margin=2.5cm}
\usepackage{amsmath}
\usepackage{amssymb}
\usepackage{listings}
\usepackage{hyperref}
\usepackage{graphicx}
\usepackage{booktabs}
\usepackage{array}

\title{Wochenbericht 09 - Praktikum "Verteilte Systeme": Abschluss MVP}
\author{Philipp Patt}
\date{\today}

\lstset{
    language=Java,                
    basicstyle=\footnotesize\ttfamily,
    numbers=left,                  
    numberstyle=\tiny\color{gray},
    stepnumber=1,                   
    numbersep=5pt,                 
    backgroundcolor=\color{white}, 
    showspaces=false,               
    showstringspaces=false,         
    showtabs=false,                
    frame=single,                 
    tabsize=4,                      
    captionpos=b,                   
    breaklines=true,                
    breakatwhitespace=true,        
    title=\lstname,               
    escapeinside={\%*}{*)},         
    morekeywords={Integer,IntWritable, Iterable, Text},
}

\begin{document}
\maketitle
\section{Mitglieder des Projektes }

\begin{tabular}{>{\raggedright\arraybackslash}p{3cm} >{\raggedright\arraybackslash}p{4cm} >{\centering\arraybackslash}p{2cm} >{\centering\arraybackslash}p{2cm} >{\raggedright\arraybackslash}p{3cm}}
\toprule
\textbf{Mitglied des Projektes} & \textbf{Aufgabe} & \textbf{Fortschritt} & \textbf{Zeiteinsatz} & \textbf{Check} \\
\midrule
Manh-An David Dao & MVP  & 80\% & 2h & in review\\
\hline
Jannik Schön & MVP  & 80\% & 4h & in review \\
\hline
Marc Siekmann & ITS-Board View und RPC Übertragung & 80\% & 5h & in progress \\
\hline
Philipp Patt & MVP & 80\% & 6h & in progress \\

\bottomrule
\end{tabular}

\section{Bearbeitete Themen und Schlüssel Erkenntnisse}

\subsection{ITS-Board View und RPC Übertragung}

Auf dem ITS-Board wird das IO des Views implementiert. Dazu wurde ein Vorlage erstellt, die noch vervollständigt werden muss\footnote{\url{https://git.haw-hamburg.de/infwgi246/vs_praktikum-2025-sose/-/merge_requests/26}}.
Für RPC wurde sich folgendes überlegt und muss noch in das arc42 Dokument der Middleware übertragen werden:\\
- Ein UDP Paket hat eine Standardgröße von 512 Byte \footnote{Van Steen, Tanenbaum(2025): Distributed Systems; 4th Edition, S. 366}. Daher wird pro Paket genau ein RPC-Aufruf durchgeführt.
Das hat den Vorteil, dass jedes einzelnes Paket als vollständige Information gilt. Das Ziel Safety ist somit mit einem Zeitstempel oder eindeutigen Token realisierbar \footnote{\url{https://www.rfc-editor.org/rfc/rfc5531\#section-5}}. Sollte der Zeitstempel bzw. Token ungültig sein, wird das Paket verworfen und dem User im UI mitgeteilt, dass ein Fehler vorliegt (Fehlertransparenz).\\
- Wir verwenden zunächst ein JSON Format, um die Daten vor dem Versenden zu Marshallen \footnote{\url{https://github.com/scimbe/vs_script/blob/main/vs-script-first-v01.pdf} S.124 Abschnitt Copy In, Copy out}. \\
- Die Namensauflösung wird mit einem zentralen Server durchgeführt. Dies wird ebenfalls mit RPC durchgeführt. Der Client wartet dann auf eine Antwort. Diese ist ebenfalls ein RPC-Aufruf vom Nameserver mittels einer Callback-Funktion, mit der IP-Adresse:Port als Antwort. Um häufige Anfragen zu vermeiden, muss gecached werden. 

\subsection{MVP Abschluss}

Der MVP ist kurz vor Abschluss \footnote{\url{https://git.haw-hamburg.de/infwgi246/vs_praktikum-2025-sose/-/merge_requests/27}}.
Der StateService, MoveAdapter und Controller wurden in einem neuen Baustein namens Core gruppiert. 
Die Kommunikation zwischen MVP-IO und Core wurde getestet. Die Anbindung der ActuatorController wurde vorerst gemockt und gilt es noch via grpc umzusetzen.


\section{Nächste Schritte}
- Gelerntes aus MVP in entsprechendes arc42 Dokument überführen (\footnote{\url{https://git.haw-hamburg.de/infwgi246/vs_praktikum-2025-sose/-/merge_requests/29}},\footnote{\url{https://git.haw-hamburg.de/infwgi246/vs_praktikum-2025-sose/-/merge_requests/28}})\\
- Einrichtung eines REDIS Namensserver mit RPC, der eine hierarchische Namensauflösung beinhaltet \\
- grpc durch eigenes rpc ersetzen \\


\end{document}
