\documentclass{article}
\usepackage[utf8]{inputenc}
\usepackage[T1]{fontenc}
\usepackage{geometry}
\geometry{a4paper, margin=2.5cm}
\usepackage{amsmath}
\usepackage{amssymb}
\usepackage{listings}
\usepackage{hyperref}
\usepackage{graphicx}
\usepackage{booktabs}
\usepackage{array}

\title{Wochenbericht I - Praktikum "Verteilte Systeme": Die Ersten Schritte}
\author{Marc Siekmann}
\date{\today}

\lstset{
    language=Java,                
    basicstyle=\footnotesize\ttfamily,
    numbers=left,                  
    numberstyle=\tiny\color{gray},
    stepnumber=1,                   
    numbersep=5pt,                 
    backgroundcolor=\color{white}, 
    showspaces=false,               
    showstringspaces=false,         
    showtabs=false,                
    frame=single,                 
    tabsize=4,                      
    captionpos=b,                   
    breaklines=true,                
    breakatwhitespace=true,        
    title=\lstname,               
    escapeinside={\%*}{*)},         
    morekeywords={Integer,IntWritable, Iterable, Text},
}

\begin{document}
\maketitle
\section{Mitglieder des Projektes }

\begin{tabular}{>{\raggedright\arraybackslash}p{3cm} >{\raggedright\arraybackslash}p{4cm} >{\centering\arraybackslash}p{2cm} >{\centering\arraybackslash}p{2cm} >{\raggedright\arraybackslash}p{3cm}}
\toprule
\textbf{Mitglied des Projektes} & \textbf{Aufgabe} & \textbf{Fortschritt} & \textbf{Zeiteinsatz} & \textbf{Check} \\
\midrule
Manh-An David Dao & Aufbau der DevOps-Toolchain, Git, Plattformen, Systemarchitektur & 0\% & ! & ! \\
\hline
Jannik Schön & Erstellung der Projektdokumentation nach arc42 & 0\% & ! & ! \\
\hline
Marc Siekmann & Einarbeitung des ITS-BRD & 0\% &  &  \\
\hline
Phillip Patt & Vorbereitung und Aufbau des Kransystems & 0\% &  & \\

\bottomrule
\end{tabular}


\section{Zusammenfassung der Woche}

In der ersten Woche fand die Gruppenfindung statt. Außerdem wurden wichtige organisatorische Grundlagen, wie ein Jour-Fixe, geplant und festgelegt. Weiterhin wurde die Aufgabe nach dem Prinzip des „Divide and Conquer“ aufgeteilt. \\\\
In der zweiten Woche wurden die Rahmenbedingungen, sowie ein grundlegendes Verständnis der vorliegenden Aufgabe erörtert. Weiterhin haben die verschiedenen Teammitglieder ihre zugeteilten Aufgaben erledigt. So wurde sich in technischer Sicht mit dem zu bedienenden Roboter und dem ITS-Board beschäftigt. Außerdem wurde ein Entwicklungsdokument, hier mithilfe des ARC-42 Templates, entworfen und eine Entwicklungspipepline eingeführt. 
\\\\
Wesentliche Pull request/commits des Projektes waren:\\
- WIP: Arc42 chapter 1 \& 2 added \\
- Protocol





\section{Bearbeitete Themen und Schlüssel Erkenntnisse}
\subsection{Aufbau des DevOps}
% TODO DAVID
\subsection{Erstellung der Dokumentation}
% TODO JANNIK
\subsection{Einarbeitung des ITS-BRD}
% TODO MARC
\subsection{Aufbau des Kransystems}
% TODO PHILLIP



\section{Bezug zur Vorlesung}
% TODO MARC 


\section{Nächste Schritte}
Die erste Iteration des Arc42-Protokolls
% TODO Phillip - Roboter Arm


\end{document}
