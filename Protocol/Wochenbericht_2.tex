\documentclass{article}
\usepackage[utf8]{inputenc}
\usepackage[T1]{fontenc}
\usepackage{geometry}
\geometry{a4paper, margin=2.5cm}
\usepackage{amsmath}
\usepackage{amssymb}
\usepackage{listings}
\usepackage{hyperref}
\usepackage{graphicx}
\usepackage{booktabs}
\usepackage{array}

\title{Wochenbericht II - Praktikum "Verteilte Systeme": Erste arc42 Iteration}
\author{TBD}
\date{\today}

\lstset{
    language=Java,                
    basicstyle=\footnotesize\ttfamily,
    numbers=left,                  
    numberstyle=\tiny\color{gray},
    stepnumber=1,                   
    numbersep=5pt,                 
    backgroundcolor=\color{white}, 
    showspaces=false,               
    showstringspaces=false,         
    showtabs=false,                
    frame=single,                 
    tabsize=4,                      
    captionpos=b,                   
    breaklines=true,                
    breakatwhitespace=true,        
    title=\lstname,               
    escapeinside={\%*}{*)},         
    morekeywords={Integer,IntWritable, Iterable, Text},
}

\begin{document}
\maketitle
\section{Mitglieder des Projektes }

\begin{tabular}{>{\raggedright\arraybackslash}p{3cm} >{\raggedright\arraybackslash}p{4cm} >{\centering\arraybackslash}p{2cm} >{\centering\arraybackslash}p{2cm} >{\raggedright\arraybackslash}p{3cm}}
\toprule
\textbf{Mitglied des Projektes} & \textbf{Aufgabe} & \textbf{Fortschritt} & \textbf{Zeiteinsatz} & \textbf{Check} \\
\midrule
Manh-An David Dao & arc42 Doku: Kapitel 1-4, 6, 11 & 0\% & 4h & ok \\
\hline
Jannik Schön & arc42 Doku: Kapitel 1-4, 5, 10 & 0\% & 4h & ok \\
\hline
Marc Siekmann & arc42 Doku: Kapitel 1-4, 8, 9 & 0\% & 4h & ok \\
\hline
Phillip Patt & arc42 Doku: Kapitel 1-4, 7; Aufbau des Roboter Repositories & 0\% & 4h & ok \\

\bottomrule
\end{tabular}


\section{Zusammenfassung der Woche}

In der zweiten Woche fand die Bearbeitung der ersten Iteration der arc42 Dokumentation statt. Wichtig war uns, dass alle das gleiche Verständnis der Aufgabe und der arc42 Dokumentation haben. Daher haben wir uns dafür entschieden, die Abschnitte 1-4 gemeinsam durchzuführen. Die nachfolgenden Kapitel haben wir, wie oben in der Tabelle beschrieben, aufgeteilt. Nach einer individuellen Einarbeitungszeit von ca. 0,5h haben wir uns die zugeteilten Abschnitte gegenseitig erklärt, sodass alle die einzelne Abschnitte inhaltlich verstanden haben. Zudem wurde geklärt in welchem Kontext der jeweilige Abschnitt im Gesamtdokument zu verstehen ist. Danach haben wir die zugeteilten Abschnitte für das bestehende Projekt ausgefüllt.
\\\\
Wesentliche Pull request/commits des Projektes waren:\\
% TODO wichtig

%- WIP: Arc42 chapter 1 \& 2 added \\
%- Protocol


\section{Bearbeitete Themen und Schlüssel Erkenntnisse}
Hinweis: Die Kapitelnummern beziehen sich auf die arc42 Dokumentation (https://arc42.de/overview/)

\subsection{arc42 Kapitel 1 - 4}
% TODO ALLE 
In Kapitel 1 - 4 werden die Anforderungen und die Umgebung in der das System betrieben wird beschrieben. In Kapitel 4 wird dann darauf aufbauend eine Lösungsstrategie der Architektur vorgestellt. Dies haben wir gemeinsam ausgefüllt. Wir haben festgestellt, dass wir noch unzureichenden Wissen in den meisten Themenfeldern bzgl. verteilte Systeme haben. Dieses Kapitel wird sich in Laufe der nächsten Wochen ausgebaut, da organisatorische Entscheidungen und Entwicklungsentscheidungen (Tools etc.) noch unklar sind. Möglicherweise entscheiden wir uns für eine tabellarische Form.


\subsection{arc42 Kapitel 6: Laufzeitsicht}
Für die erste Iteration der Architekturbeschreibung nach arc42 verwenden wir bewusst Sequenzdiagramme, da sie sich besonders gut zur Darstellung der zeitlichen Abläufe und Interaktionen zwischen Systemkomponenten eignen – klar, verständlich und ohne unnötige Komplexität.\\\\
Andere UML-Diagrammarten wie Aktivitäts- oder Zustandsdiagramme wurden in dieser Phase bewusst ausgelassen, da sie für die grundlegende Kommunikation noch nicht erforderlich sind. Ziel ist ein einfaches, aber aussagekräftiges Bild des Systemverhaltens, das später erweitert werden kann.

\subsection{arc42 Kapitel 11: Risiken}
Da sich unser Architekturansatz noch in der frühen Phase befindet, lassen sich viele Risiken aktuell noch nicht konkret benennen.
Wir konzentrieren uns daher zunächst auf technische Risiken, die sich direkt aus der gewählten Systemstruktur ergeben. Weitere Risiken werden in späteren Iterationen ergänzt, sobald zentrale Entscheidungen getroffen wurden.
% TODO DAVID



\subsection{arc42 Kapitel 5 und 10}
% TODO JANNIK


% TODO MARC
\subsection{arc42 Kapitel 8: Konzepte}
Das Kapitel wurde als Zusammenfassung des Projektes verstanden. Somit wurden einige Oberthemen rausgesucht und anschliessend die verwendeten Konzepte beschrieben. Dies muss in den nächsten Iterationen ergänzt und verbessert werden.

\subsection{arc42 Kapitel 9: Architekturentscheidungen}
Das Kapitel  beschreibt die Architektur Entscheidungen. Wir haben uns darauf geeinigt unsere Entscheidungen dort tabellarisch festzuhalten und mit einem Zeitstempel (Wochenzahl) zu versehen. So können wir nachschauen, wann wir eine Entscheidung getroffen haben und eine Begründung dazu. Das ergibt am Ende einen Zeitverlauf, sodass wir Architekturentscheidungen nachvollziehen können


\subsection{arc42 Kapitel 7 und Roboter Repository}
% TODO PHILLIP



\section{Bezug zur Vorlesung}
% TODO ALLE 

 

\section{Nächste Schritte}
Da in dieser Woche die erste Iteration des arc42 Dokuments durchgeführt wurde, kann in den folgenden Wochen die Dokumentation iterativ aktualisiert werden. Zunächst haben wir aber festgestellt die Anforderungen strukturierte aufgeschrieben und vervollständigt werden müssen. Die Aktualisierung des arc42 Dokuments wird sonst nach der Vorlesung geschehen, da zu diesem Zeitpunkt mit den meisten neuen Erkenntnissen gerechnet wird. Weiterhin kann sich von nun an mit dem ITS-Board, Roboterarm, Toolchain und den weiteren anstehenden Aufgaben beschäftigt werden. Es wurde sich zunächst darauf geeinigt, eine Grundfunktionalität herzustellen, um die aktuelle Iteration testen zu können. Dabei könnte ein MVP helfen, der pro Iteration ergänzt wird. Die Aufgabenteilung dazu wird noch diskutiert.



\end{document}
