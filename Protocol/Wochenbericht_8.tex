\documentclass{article}
\usepackage[utf8]{inputenc}
\usepackage[T1]{fontenc}
\usepackage{geometry}
\geometry{a4paper, margin=2.5cm}
\usepackage{amsmath}
\usepackage{amssymb}
\usepackage{listings}
\usepackage{hyperref}
\usepackage{graphicx}
\usepackage{booktabs}
\usepackage{array}

\title{Wochenbericht 08 - Praktikum "Verteilte Systeme": Überarbeitung Applikation}
\author{David Dao}
\date{\today}

\lstset{
    language=Java,                
    basicstyle=\footnotesize\ttfamily,
    numbers=left,                  
    numberstyle=\tiny\color{gray},
    stepnumber=1,                   
    numbersep=5pt,                 
    backgroundcolor=\color{white}, 
    showspaces=false,               
    showstringspaces=false,         
    showtabs=false,                
    frame=single,                 
    tabsize=4,                      
    captionpos=b,                   
    breaklines=true,                
    breakatwhitespace=true,        
    title=\lstname,               
    escapeinside={\%*}{*)},         
    morekeywords={Integer,IntWritable, Iterable, Text},
}

\begin{document}
\maketitle
\section{Mitglieder des Projektes }

\begin{tabular}{>{\raggedright\arraybackslash}p{3cm} >{\raggedright\arraybackslash}p{4cm} >{\centering\arraybackslash}p{2cm} >{\centering\arraybackslash}p{2cm} >{\raggedright\arraybackslash}p{3cm}}
\toprule
\textbf{Mitglied des Projektes} & \textbf{Aufgabe} & \textbf{Fortschritt} & \textbf{Zeiteinsatz} & \textbf{Check} \\
\midrule
Manh-An David Dao & Lösungsstrategie Applikation überarbeitet & 80\% & 2h & in review\\
\hline
Jannik Schön & Lösungsstrategie Applikation überarbeitet & 80\% & 2h & in review \\
\hline
Marc Siekmann & Implementierung MVP View & 70\% & 4h & in progress \\
\hline
Philipp Patt & Embedded RPC \& Platzhalter IO & 60\% & 4h & in progress \\

\bottomrule
\end{tabular}



\section{Zusammenfassung der Woche}

In dieser Woche fand die Bearbeitung der Lösungsstrategie der Applikation statt, außerdem wurde das MVP der View implementiert.
\\\\
Wesentliche pull-requests/commits sind in den Fußnoten hinterlegt. \\ \\


\section{Bearbeitete Themen und Schlüssel Erkenntnisse}

\subsection{Lösungsstrategie Applikation überarbeitet}
Die Lösungsstrategie der Applikation wurde überarbeitet. Dabei wurden allen bekannten Methoden von dem Model, View und dem Controller beschrieben\footnote{\url{https://git.haw-hamburg.de/infwgi246/vs_praktikum-2025-sose/-/blob/arc42/application/chapter_4/update/Docs/loesungsstrategie.tex}}. 
Das Model speichert dabei den ausgewählten Roboter und die anderen verfügbaren Roboter im System. Das Model besitzt dabei Methoden, um Beobachter hinzuzufügen, zu entfernen und zu benachrichtigen\footnote{\url{https://github.com/scimbe/vs_script/blob/main/vs-script-first-v01.pdf} S.91}. 
Dafür wird ein Observable implementiert, um ressourcen zu schonen, da Polling vermieden wird.  
Das Model implementiert außerdem die Geschäftslogik, so dass keine ungültigen Roboter ausgewählt werden können. Außerdem wird der Controller benachrichtigt, falls es zu einer Zustandsänderung kommt.
\\\\
Der Controller folgt dem Observer-Pattern, bei dem er Änderungen vom Modell empfängt und die View aktualisiert. 
Das Observer-Pattern sorgt für eine lose Kopplung zwischen den Komponenten und fördert die Modularität.
\footnote{\url{https://github.com/scimbe/vs_script/blob/main/vs-script-first-v01.pdf} S.98 f.}.
Er ist stateless, der Nachrichtenfluss ist unidirektional, und der Controller verarbeitet keine Eingaben von der View.
Stateless fördert Austauschbarkeit und Fehlerreduzierung durch reduzierte Komplexität\footnote{\url{https://github.com/scimbe/vs_script/blob/main/vs-script-first-v01.pdf} S.179}.
Unidirektional sorgt für Klarheit und Einfachheit, indem es die Verantwortlichkeiten klar trennt.

\subsection{Implementierung MVP View}
Das View besteht aus den IO und UI. Auf dem ITS-Board wird das IO implementiert. Der Block IO besteht aus einer Funktion, die die Eingabe überwacht. Die Eingabe löst einen Funktionsaufruf aus.
Daher wurde auf dem ITS-Board ein UDP-Server implementiert. Es wurde sich für das Transportprotokoll UDP entschieden, da uns in der Middleware schnelle Übertragungen \footnote{\url{https://github.com/scimbe/vs_script/blob/main/vs-script-first-v01.pdf} S.173 Abs. 2}, in Bezug auf die
geforderte Safety und Echtzeitfähigkeit, als wichtig eingestuft wird. Für die Delayoptimierung muss HTTP/3 oder QUIC auf Umsetzung geprüft werden\footnote{\url{https://github.com/scimbe/vs_script/blob/main/vs-script-first-v01.pdf} S.173 Abs. 3}.
Um die Funktionalität zu testen wird zunächst jeder Funktionsaufruf mit einem, an das Richard Maturity Model Level 0 angelegte, RPC-Stil übertragen \footnote{\url{https://github.com/scimbe/vs_script/blob/main/vs-script-first-v01.pdf} S.181 ff.}. Dies muss noch ausgearbeitet werden, da hierbei noch kein HTTP Standard verwendet wird, sondern nur der Aufruf als String.
Da das ITS-Board ein Embedded System ist, wurde hauptsächlich die Zeit in das Versenden und Empfangen von UDP-Paketen getestet und erfolgreich durchgeführt.
\footnote{\url{https://git.haw-hamburg.de/infwgi246/vs_praktikum-2025-sose/-/tree/mvp/view/Client/ITS-Board}} 

\subsection{Embedded RPC \& Platzhalter IO}
Beim erstellen des MVP mithilfe von grpc sind im embedded Kontext Hürden aufgefallen. Um nicht zu viel Zeit mit der temporären Lösung grpc zu verschwenden wurde begonnen ein minimaler IO Baustein auf potenterer Harware umzusetzen, wo grpc ohne weiteres möglich ist. 
Dies ermöglicht uns den MVP funktional zu vervollständigen und die nächsten Schritte zu beginnen. 
 
\section{Nächste Schritte}
- Vollständige Implementierung der View auf dem ITS-Board \\
- Die Bausteine des Models implementieren \\
- Einrichtung eines REDIS Namensserver, der eine hierarchische Namensauflösung beinhaltet \\
- MVP Komponent miteinander verbinden und funktional testen \\
- Gelerntes aus MVP in arc42 Dokument überführen

\end{document}
