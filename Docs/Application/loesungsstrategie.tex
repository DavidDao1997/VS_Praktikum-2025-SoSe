\chapter{Lösungsstrategie}

\section{Controller}
Der Controller fungiert als Observer. 
Er empfängt Benachrichtigungen vom Modell, wenn Änderungen oder Aktualisierungen erfolgen, und leitet diese Informationen an die View weiter, um sie zu aktualisieren. 
Der Controller speichert keine Zustände und sendet keine Eingaben von der View an das Modell.

%Eventuell später parametertype hinzufügen und genauer definieren.
\begin{table}[h!]
    \centering
    \begin{tabular}{|p{5cm}|p{5cm}|p{5cm}|}
        \hline
        \textbf{Funktion} & \textbf{Voraussetzung} & \textbf{Semantik} \\
        \hline
        void update(AvailableRobots: String[], SelectedRobotIdx: int, Error: bool, Confirmation: bool) & Modell hat Änderungen & Benachrichtigt den Controller, der die View aktualisiert. \\
        \hline
    \end{tabular}
    \caption{Funktionen des Controllers}
    \label{tab:Controller}
\end{table}


\section{Modell}
\begin{table}[h!]
    \centering
    \begin{tabular}{|p{5cm}|p{5cm}|p{5cm}|}
        \hline
        \textbf{Funktion} & \textbf{Voraussetzung} & \textbf{Semantik} \\
        \hline
        ActuatorControlWrapper(Host: String, Port: int, Actuator: String) & Gültige Netzwerkverbindung & Initialisiert ein Aktuator mit Host, Port. \\
        \hline
        Void move(Enum Robotdirection) & Verbindung zwischen IO und MoveAdapter besteht &  Gibt an in welche Richtung sich der Roboter bewegen soll. \\
        \hline
        Void move(Enum ActuatorDirection ) & Verbindung zwischen MoveAdapter und ActuatorController besteht & Erhöht oder verringert den Steuerwert um 1 und ruft applyValue() auf. \\
        \hline
        Void applyValue() & Aktuator gültig; Wert im Bereich [0,100] & Wendet den aktuellen Wert auf den Aktuator an. \\
		\hline
		Void notify() & StateService initialisiert & Teilt Änderungen im Modell den Listener mit \\
        \hline
        Void addListener() & Listener gültig & Fügt einen Listener hinzu, der bei Zustandsänderung benachrichtigt wird. \\
        \hline
        Void removeListener() & Listener gültig & Entfernt den angegebenen Listener. \\
        \hline
		Int register(ActuatorName) & StateService initialisiert, Verbindung zwischen ActuatorController und StateService besteht& registriert einen ActuatorController beim StateService \\
		\hline
		setError(boolean, boolean)& & \\
		\hline
        Void setSelectedRobot(Enum SelectDirection) & Roboter in availableRobots & Setzt den aktuellen Roboter und benachrichtigt alle Listener. \\
        \hline
        String getSelectedRobot() & StateService initialisiert & Gibt den aktuell ausgewählten Roboter zurück. \\
        \hline
        Int getValue() & ActuatorController initialisiert & Gibt den aktuellen Steuerwert zurück. \\
        \hline
        Void addAvailableRobot(String) & StateService initialisiert, Gültige Roboter-ID & Fügt einen Roboter zur Liste verfügbarer Roboter hinzu. \\
        \hline
        Void removeAvailableRobot(String) & StateService initialisiert, Gültige Roboter-ID & Entfernt einen Roboter aus der Liste verfügbarer Roboter. \\
        \hline
        String[] getAvailableRobots() & StateService initialisiert & Gibt alle bekannten Roboter zurück. \\
        \hline
    \end{tabular}
    \caption{Funktionen des Modells}
    \label{tab:Methodenbeschreibung}
\end{table}

\clearpage
\section{View}
Die View besteht aus den unabhängigen Blöcken IO und UI. Die UI bietet eine Softwareschnittstelle an, die IO keine.

\subsection{IO Funktionen}
\begin{table}[h!]
    \centering
    \begin{tabular}{|p{5cm}|p{5cm}|p{5cm}|}
        \hline
        \textbf{Funktion} & \textbf{Voraussetzung} & \textbf{Semantik} \\
        \hline
        void readInputs() & Eingabe durch den Benutzer erfolgt & Überprüft die Benutzereingaben und löst entsprechende Aktionen aus. \\
        \hline
        int initIO() & IO-Hardware verfügbar & Initialisiert und überprüft die IO-Hardwareschnittstellen und gibt einen Fehlercode bei Problemen zurück. \\
        \hline
    \end{tabular}
    \caption{IO Funktionen}
    \label{tab:IOFunktionen}
\end{table}

\subsection{UI Funktionen}
\begin{table}[h!]
    \centering
    \begin{tabular}{|p{5cm}|p{5cm}|p{5cm}|}
        \hline
        \textbf{Funktion} & \textbf{Voraussetzung} & \textbf{Semantik} \\
        \hline
        void updateView(AvailableRobots: String[],SelectedRobotIdx: int, Error: bool, Confirmation: bool) & Gültige Modell-Daten vorhanden & View-Schnittstelle. Aktualisiert die UI mit den neuesten Roboter-Daten und Statusinformationen (Fehler, Bestätigung). \\ 
        \hline
    \end{tabular}
    \caption{UI Funktionen}
    \label{tab:UIFunktionen}
\end{table}


