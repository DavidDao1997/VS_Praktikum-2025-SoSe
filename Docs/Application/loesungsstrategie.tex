\chapter{Lösungsstrategie}

\section{Controller}
Der Controller fungiert als Observer. 
Er empfängt Benachrichtigungen vom Modell, wenn Änderungen oder Aktualisierungen erfolgen, und leitet diese Informationen an die View weiter, um sie zu aktualisieren. 
Der Controller speichert keine Zustände und sendet keine Eingaben von der View an das Modell.

%Eventuell später parametertype hinzufügen und genauer definieren.
\begin{table}[h!]
    \centering
    \begin{tabular}{|p{5cm}|p{5cm}|p{5cm}|}
        \hline
        \textbf{Funktion} & \textbf{Voraussetzung} & \textbf{Semantik} \\
        \hline
        void update(AvailableRobots: String[], SelectedRobotIdx: int, Error: bool, Confirmation: bool) & Modell hat Änderungen & Benachrichtigt den Controller, der die View aktualisiert. \\
        \hline
    \end{tabular}
    \caption{Funktionen des Controllers}
    \label{tab:Controller}
\end{table}


\section{Model}

    \subsection{StateService}
    \begin{table}[h!]
        \centering
        \begin{tabular}{|p{5cm}|p{5cm}|p{5cm}|}
            \hline
            \textbf{Funktion} & \textbf{Voraussetzung} & \textbf{Semantik} \\
            \hline
            void select(SelectDirection selectDirection) & keine & Wählt einen Roboter aus \\
            \hline
            void setError(Boolean error) & Fehler ist aufgetreten & Setzt den Fehlerstatus. \textbf{(Beschreibung ergänzen)} \\
            \hline
            %void register(ActuatorName actuator) & Verbindung zwischen ActuatorController und StateService besteht & Registriert einen ActuatorController beim StateService. \\
            %\hline
            %void heartbeat(ActuatorName actuator) & In regelmäßigen Zeitfenstern & Benachrichtigt den StateService über den aktuellen Status der Aktoren.
            %\hline
            void reportHealth(String [] actuatorNames) & In regelmäßigen Zeitfenstern und wenn subscribed wurde & Benachrichtigt den StateService über den aktuellen Gesundheitsstatus der Aktoren. \\
            \hline
            private void notify() & StateService erfasst für Controller relevante Änderung & Benachrichtigt den Controller über Änderungen im StateServices. \\
            \hline
        \end{tabular}
        \caption{Funktionen des StateService}
        \label{tab:StateService}
    \end{table}


\clearpage
    \subsection{MoveAdapter}
    \begin{table}[h!]
        \centering
        \begin{tabular}{|p{5cm}|p{5cm}|p{5cm}|}
            \hline
            \textbf{Funktion} & \textbf{Voraussetzung} & \textbf{Semantik} \\
            \hline
            void move(Enum RobotDirection) & Verbindung zwischen IO und MoveAdapter besteht & Gibt die Richtung an, in die sich der Roboter bewegen soll. \\
            \hline
            void setSelectedRobot(String robotName) & In regelmäßigen Zeitfenstern & Setzt den aktuell ausgewählten Roboter.\\
            \hline
        \end{tabular}
        \caption{Funktionen des MoveAdapter}
        \label{tab:MoveAdapter}
    \end{table}

    \subsection{ActuatorController}
    \begin{table}[h!]
        \centering
        \begin{tabular}{|p{5cm}|p{5cm}|p{5cm}|}
            \hline
            \textbf{Funktion} & \textbf{Voraussetzung} & \textbf{Semantik} \\
            \hline
            void move(Enum ActuatorDirection) & Verbindung zwischen MoveAdapter und ActuatorController besteht & Erhöht oder verringert den Steuerwert um 1 und ruft \texttt{applyValue()} auf. \\
            \hline
            %void ActuatorControlWrapper(Host: String, Port: int, Actuator: String) & Gültige Netzwerkverbindung & Initialisiert einen Aktuator mit dem angegebenen Host und Port. \\
            %\hline
        \end{tabular}
        \caption{Funktionen des ActuatorController}
        \label{tab:ActuatorController}
    \end{table}

    \subsection{Watchdog}
    \begin{table}[h!]
        \centering
        \begin{tabular}{|p{5cm}|p{5cm}|p{5cm}|}
            \hline
            \textbf{Funktion} & \textbf{Voraussetzung} & \textbf{Semantik} \\
            \hline
            void checkIn(String serviceName) & service ist verfügbar & Sendet ein Signal an den Watchdog, um die Verfügbarkeit zu kommunizieren. \\
            \hline
            void subscribe(String serviceName) & Die Verfügbarkeit eines Services soll kommuniziert werden & Abonniert den Watchdog, um Benachrichtigungen über den Gesundheitsstatus zu erhalten. \\
            \hline
        \end{tabular}
        \caption{Watchdog Funktionen}
        \label{tab:WatchdogFunktionen}
    \end{table}





    %\begin{table}[h!]
        %\centering
        %\begin{tabular}{|p{5cm}|p{5cm}|p{5cm}|}
        %    \hline
       %     \textbf{Funktion} & \textbf{Voraussetzung} & \textbf{Semantik} \\
            %\hline
            %ActuatorControlWrapper(Host: String, Port: int, Actuator: String) & Gültige Netzwerkverbindung & Initialisiert einen Aktuator mit dem angegebenen Host und Port. \\
            %\hline
            %void move(Enum RobotDirection) & Verbindung zwischen IO und MoveAdapter besteht & Gibt die Richtung an, in die sich der Roboter bewegen soll. \\
            %\hline
            %void move(Enum ActuatorDirection) & Verbindung zwischen MoveAdapter und ActuatorController besteht & Erhöht oder verringert den Steuerwert um 1 und ruft \texttt{applyValue()} auf. \\
            %\hline
            %%void applyValue() & Aktuator gültig; Wert im Bereich [0,100] & Wendet den aktuellen Steuerwert auf den Aktuator an. \\
            %\hline
            %void notify() & StateService initialisiert & Benachrichtigt alle Listener über Änderungen im Modell. \\
            %\hline
            %void addListener() & Listener gültig & Fügt einen Listener hinzu, der bei Zustandsänderungen benachrichtigt wird. \\
            %\hline
            %void removeListener() & Listener gültig & Entfernt den angegebenen Listener. \\
            %\hline
            %void register(ActuatorName) & StateService initialisiert, Verbindung zwischen ActuatorController und StateService besteht & Registriert einen ActuatorController beim StateService. \\
            %\hline
            %void setError(boolean, boolean) & -- & Setzt den Fehlerstatus. \textbf{(Beschreibung ergänzen)} \\
            %\hline
            %void setSelectedRobot(Enum SelectDirection) & Roboter in \texttt{availableRobots} vorhanden & Setzt den aktuell ausgewählten Roboter und benachrichtigt alle Listener. \\
            %\hline
            %String getSelectedRobot() & StateService initialisiert & Gibt den aktuell ausgewählten Roboter zurück. \\
            %\hline
            %int getValue() & ActuatorController initialisiert & Gibt den aktuellen Steuerwert zurück. \\ Muss updateValue() sein, aber wie?
            %\hline
            %void addAvailableRobot(String []) & StateService initialisiert, gültige Roboter-ID & Fügt einen Roboter zur \textbf{internen Liste} (\texttt{ArrayList<String>}) verfügbarer Roboter hinzu. \\
            %\hline
            %void removeAvailableRobot(String) & StateService initialisiert, gültige Roboter-ID & Entfernt einen Roboter aus der \textbf{internen Liste} (\texttt{ArrayList<String>}) verfügbarer Roboter. \\
            %\hline
            %String[] getAvailableRobots() & StateService initialisiert & Gibt alle bekannten Roboter zurück, die in der \textbf{internen Liste} (\texttt{ArrayList<String>}) gespeichert sind. \\
            %\hline
       % \end{tabular}
      %  \caption{Funktionen des Modells}
     %   \label{tab:Methodenbeschreibung}
    %\end{table}


\clearpage
\section{View}
Die View besteht aus den unabhängigen Blöcken IO und UI. Die UI bietet eine Softwareschnittstelle an, die IO keine.

\subsection{IO Funktionen}
\begin{table}[h!]
    \centering
    \begin{tabular}{|p{5cm}|p{5cm}|p{5cm}|}
        \hline
        \textbf{Funktion} & \textbf{Voraussetzung} & \textbf{Semantik} \\
        \hline
        void readInputs() & Eingabe durch den Benutzer erfolgt & Überprüft die Benutzereingaben und löst entsprechende Aktionen aus. \\
        \hline
        int initIO() & IO-Hardware verfügbar & Initialisiert und überprüft die IO-Hardwareschnittstellen und gibt einen Fehlercode bei Problemen zurück. \\
        \hline
    \end{tabular}
    \caption{IO Funktionen}
    \label{tab:IOFunktionen}
\end{table}

\subsection{UI Funktionen}
\begin{table}[h!]
    \centering
    \begin{tabular}{|p{5cm}|p{5cm}|p{5cm}|}
        \hline
        \textbf{Funktion} & \textbf{Voraussetzung} & \textbf{Semantik} \\
        \hline
        void updateView(AvailableRobots: String[],SelectedRobotIdx: int, Error: bool, Confirmation: bool) & Gültige Modell-Daten vorhanden & View-Schnittstelle. Aktualisiert die UI mit den neuesten Roboter-Daten und Statusinformationen (Fehler, Bestätigung). \\ 
        \hline
    \end{tabular}
    \caption{UI Funktionen}
    \label{tab:UIFunktionen}
\end{table}
\clearpage



