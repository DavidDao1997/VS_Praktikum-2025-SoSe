\chapter{Verteilungssicht}

\begin{figure}[h!]
	\centering
	\includegraphics[scale=.5]{diagrams/deploymentView.png}
	\caption{Deployment Diagramm (Produtiv)}
	\label{fig:deployment-prod-grafik}
\end{figure}

\section{Begründung}

Die Verteilung der Software-Bausteine auf verschiedene Hardware-Komponenten (PC, ITS-BRD, Raspberry Pi, Cloud) wurde gewählt, um folgende Systemanforderungen zu erfüllen:

\begin{itemize}
    \item \textbf{Echtzeitnahe Verarbeitung:} Die Verwendung von UDP ermöglicht eine schnelle, verbindungslose Übertragung von Sensor- und Steuerdaten mit minimaler Latenz.
    \item \textbf{Lastverteilung:} Durch die Verlagerung von Steuerungs- und Zustandsdiensten in die Cloud wird die Last von den lokalen Geräten genommen und die Systemarchitektur bleibt flexibel und erweiterbar.
    \item \textbf{Hardware-Nähe:} Die IO-Module sowie der ActuatorController laufen auf Geräten in unmittelbarer Nähe zu den physischen Sensoren und Aktoren, um direkte Ansteuerung ohne zusätzliche Kommunikationsverzögerung zu ermöglichen.
    \item \textbf{Flexibilität:} Die Nutzung von RPC ermöglicht modulare, verteilte Anwendungsarchitekturen, bei denen die Services flexibel auf verschiedenen Knoten bereitgestellt werden können.
\end{itemize}

\section{Qualitäts- und Leistungsmerkmale}

\begin{table}[h!]
\centering
\begin{tabular}{|l|p{10cm}|}
\hline
\textbf{Merkmal} & \textbf{Beschreibung} \\ \hline
Latenz & Geringe Latenz durch den Einsatz des verbindungslosen UDP-Protokolls. \\ \hline
Verfügbarkeit & Die Cloud-Dienste können redundant und hochverfügbar ausgelegt werden. \\ \hline
Zuverlässigkeit & UDP bietet keine Garantie für die Paketübertragung. Fehlerbehandlung und Wiederholungsmechanismen müssen von der Anwendung sichergestellt werden. \\ \hline
Echtzeitfähigkeit & Die Übertragung über UDP ermöglicht schnelle Reaktionszeiten bei der Steuerung, ist jedoch aufgrund der Verbindungsfreiheit potentiell paketverlustanfällig. \\ \hline
Skalierbarkeit & Die Cloud-Komponenten sind horizontal skalierbar und können bei steigendem Datenaufkommen erweitert werden. \\ \hline
\end{tabular}
\caption{Qualitäts- und Leistungsmerkmale der Verteilungssicht}
\end{table}

\section{Zuordnung von Bausteinen zu Infrastruktur}

\begin{table}[h!]
\centering
\begin{tabular}{|l|l|}
\hline
\textbf{Software-Baustein} & \textbf{Hardware-Komponente} \\ \hline
UI & PC/Rechner \\ \hline
IO & ITS-BRD \\ \hline
ActuatorController & Raspberry Pi \\ \hline
StateService & Cloud \\ \hline
MoveAdapter & Cloud \\ \hline
Controller & Cloud \\ \hline
\end{tabular}
\caption{Zuordnung der Software-Bausteine zu den Infrastrukturkomponenten}
\end{table}

% Notizen zu https://docs.arc42.org/section-7/

% - Geografische Orte
% - Umgebungen (dev / test / prod / temporary or ephemeral)
% - Computers (entwickler laptop/labor-pc/roboter+rasPi/ITS-board/ICC/VMs oder Container?)

% - Kein plan: Processorcs, channels, net topologies 
%   Rede ist auch manchmal von Arc42 für Hardware Desing, nehme an Processors ist also latte.
% - Mapping von Software (5. Building blocks) auf infrastruktur
% - Wir muessen nur Infrastruktur dokumentieren die für Software notwendig ist.

% Maybe the highest level deployment diagram is already contained in section 3.2. as technical context with your own 
% infrastructure as ONE black box. In this section you will zoom into this black box using additional deployment 
% diagrams.

% UML Wenn infrstruktur einigermassen komplex

% Hardware anforderungen (bspw. Java umgebung für entwickle laptop)

% Aus https://docs.arc42.org/tips/7-1/
% Weitere Hardware wie Switch/Router

% Aus https://docs.arc42.org/tips/7-2/
% If hardware plays an important role in the architecture, you can even use a node-template for that purpose, 
% similar to the following table:
% Node <node-name>
% | Responsibility	            | what is the role of this hardware element, what’s it doing?               |
% | (technical) characteristics	| i.e. nr of cpus/cores, memory, throughput, nr-of-ports, vendor, model…    |
% | associated building blocks 	| what part of the software is running on this hardware?                    |
% | reason for selection	    | why was this particular hardware selected?                                |

% https://docs.arc42.org/tips/7-3/ | Hierarchisch immer detaillierter UML standard(?) "O-O Notation"
% https://docs.arc42.org/tips/7-4/ | Hierarchisch immer detaillierter nur anders (nicht UML glaub ich)
% https://docs.arc42.org/tips/7-5/ | Mapping Building Blocks auf Hardwareumgebung(en)
% https://docs.arc42.org/tips/7-6/ | UML Hardware/Building Blocks
% https://docs.arc42.org/tips/7-7/ | Tabelle statt UML (aber auch UML)
% gibt noch 3 weitere tipps goaub ich...

% 7.1 Infrastructure Level 1

% Describe (usually in a combination of diagrams, tables, and text):

% the distribution of your system to multiple locations, environments, computers, processors, .. as well as the physical connections between them
% important justification or motivation for this deployment structure
% Quality and/or performance features of the infrastructure
% the mapping of software artifacts (building blocks) to elements of the infrastructure
% For multiple environments or alternative deployments please copy that section of arc42 for all relevant environments. **

% < insert infrastructure overview diagram >

% Motivation
% < insert description of motivation or explanation in text form>

% (optional) Quality and/or Performance Features
% < optionally insert description quality or performance features >

% Mapping
% < insert description of mapping of building blocks >

% 7.2 Infrastructure Level 2
% Here you can include the internal structure of (some) infrastructure elements from infrastructure level 1.
% Please copy the structure from level 1 for each selected element.

% 7.2.1 < Infrastructure element 1>
% < insert diagram + explanation >

% 7.2.2 < Infrastructure element 1>
% < insert diagram + explanation >

% …
% < insert diagram + explanation >

% 7.2.n < Infrastructure element 1>
% < insert diagram + explanation >