% \printglossaries

% \newglossaryentry{Architekturdokumentation}
% {
% 	name=Architekturdokumentation,
% 	description={ist ein seltenes Dokument, dass in gewissen Softwareprojekten gelegentlich auftaucht.}
% }
\chapter{Glossar}

\begin{table}[h!]
	\centering
	\begin{tabular}{p{4cm}|p{10cm}|}
		\hline
		\textbf{Begriff} & \textbf{Definition Erklährung} \\
		\hline
		GUI-Software & 
            Überbegriff für Steuerungssoftware, die mit Hilfe des LCD Displays dem Nutzer die Steuerung der Roboter ermöglicht.
            Es handlet sich hierbei um ein Programm/Executable \\
		\hline
            Roboter-Software & 
            Überbegriff für Software, die für die direkte Steuerung eines Roboters zuständig ist.
            Es handlet sich hierbei um ein Programm/Executable \\
		\hline
            Roboter/Roboterarm &
            Hardware, umfasst den Arm selbst sowie das entsprechende RasberryPi \\
		& \\
		\hline
	\end{tabular}
	\caption{Glossar}
	\label{tab:glossar}
\end{table}