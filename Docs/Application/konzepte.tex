\chapter{Konzepte}
\begin{quote}
	
\end{quote}

%\section{Offenheit}
%\begin{quote}
	
%\end{quote}
	
%\section{Verteilungstranzparenzen}
%\begin{quote}
	
%\end{quote}

%\section{Kohärenz}
%\begin{quote}
%Kohärenz wird dadurch sichergestellt, dass die Steuerung alle Roboter ausschliesslich über das ITS-Board zulässig ist.
%\end{quote}
\section{Sicherheit (Safety)}
\begin{quote}
Für jede Tastenbetätigung des Nutzers wird nun ein Bewegungsbefehl an den Roboter weitergegeben.
Dabei ist auch das gedrückthalten der Taste nur ein Bewegungsbefehl. Gleichzeitiges betätigen der Taster funktioniert nicht. 
%Watchdogs auf dem Raspberry Pi des Roboterarms überwacht die eigene Verfügbarkeit. Wenn sich der Roboterarm durch einen vorherigen Aufruf vom ITS-Board bewegt, muss eine Verbindung zum ITS-Board bestehen. Bricht diese ab, oder das Netzwerk ist zu stark ausgelastet, muss der Roboterarm sofort anhalten. Wenn das ITS-Board einen Abbruch der Verbindung feststellt ist der Roboterarm als "nicht verfügbar" erkennbar.
%Security ist keine Anforderung an das System.
\end{quote}
\section{Bedienoberfläche}
\begin{quote}
	Die Bedienung wird textuell auf dem Display des ITS-Boards dargestellt. Es wird auf Einfachheit und Eindeutigkeit geachtet. Außerdem werden die verfügbaren, sowie der ausgewählte Roboter in einem Webbrowser dargestellt.
	Auch werden Fehler, sowie Bestätigungen visuell in dem Webbrowser dargestellt.
	
\end{quote}

%\section{Fachliche Strukturen (Domäne)}

%\begin{quote}
%	Fachliche Modelle, Domänenmodelle, Business-Modelle – sie alle beschreiben Strukturen der reinen Fachlichkeit, also ohne Bezug zur Implementierungs- oder Lösungstechnologie.
%	Oftmals tauchen Teile solcher fachlichen Modelle an vielen Stellen in der Architektur, insbesondere der Bausteinsicht, wieder auf. 
%	Das "Domain-Driven-Design" oder die uralte "essentielle Systemanalyse" können Ihnen hierbei helfen.
	
%	Mit Absicht stellen wir diesen Abschnitt an den Anfang der übergreifenden Konzepte.
%\end{quote}

%\section{Typische Muster und Strukturen}

%\begin{quote}
%	Oftmals tauchen einige typische Lösungsstrukturen oder Grundmuster an mehren Stellen der Architektur auf. Beispiele dafür sind die Abhängigkeiten zwischen Persistenzschicht, Applikation sowie die Anbindung grafischer Oberflächen an die Fach- oder Domänenobjekte. Solche wiederkehrenden Strukturen beschreiben Sie möglichst nur ein einziges Mal, um Redundanzen zu vermeiden. Dieser Abschnitt erfüllt genau diesen Zweck.
%\end{quote}

\section{Ablaufsteuerung}

\begin{quote}
	Ablaufsteuerung von IT-Systemen bezieht sich sowohl auf die an der (grafischen) Oberfläche sichtbaren Abläufe als auch auf die Steuerung der Hintergrundaktivitäten. Zur Ablaufsteuerung gehört daher unter anderem die Steuerung der Benutzungsoberfläche, die Workflow- oder Geschäftsprozessteuerung sowie Steuerung von Batchabläufen.
\end{quote}



\begin{quote}
%	Wie werden Programmfehler und Ausnahmen systematisch und konsistent behandelt?
%	Wie kann das System nach einem Fehler wieder in einen konsistenten Zustand gelangen? Geschieht dies automatisch oder ist manueller Eingriff erforderlich? Dieser Aspekt hat mit Logging, Protokollierung und Tracing zu tun.
%	Welche Art Ausnahmen und Fehler behandelt ihr System? Welche Art Ausnahmen werden an welche Außenschnittstelle weitergeleitet und welche Ausnahmen behandelt das System komplett intern? Wie nutzen Sie die Exception-Handling Mechanismen ihrer Programmiersprache? Verwenden Sie checked- oder unchecked-Exceptions?
\end{quote}



%\section{Management und Administrierbarkeit}
%
%\begin{quote}
%	Größere IT-Systeme laufen häufig in kontrollierten Ablaufumgebungen (Rechenzentren) unter der Kontrolle von Operatoren oder Administratoren ab. Diese Stakeholder benötigen einerseits spezifische Informationen über den Zustand der Programme zur Laufzeit, andererseits auch spezielle Eingriffs- oder Konfigurationsmöglichkeiten.
%\end{quote}

%\section{Migration}

%\begin{quote}
%	Für manche Systeme gibt es existierende Altsysteme, die durch die neuen Systeme abgelöst werden sollen. Denken Sie als Architekt rechtzeitig auch an alle organisatorischen und technischen Aspekte, die zur Einführung oder Migration der Architektur beachtet werden müssen.
%	Beispiele:
%	\begin{itemize}
%		\item Konzept, Vorgehensweise oder Werkzeuge zur Datenübernahme und initialen Befüllung mit Daten
%		\item Konzept zur Systemeinführung oder zeitweiliger Parallelbetrieb von Alt- und Neusystem
%	\end{itemize}
%	Müssen Sie bestehende Daten migrieren? Wie führen Sie die benötigten syntaktischen oder semantischern Transformationen durch?
%\end{quote}

%\section{Parallelisierung / Nebenläufigkeit}

%\begin{quote}
%	Programme können in parallelen Prozessen oder Threads ablaufen - was die Notwendigkeit von Synchronisationspunkten mit sich bringt. Die Grundlagen dieses Aspekten legt die Parallelverarbeitung. Für die Architektur und Implementierung nebenläufiger Systeme sind viele technische Detailaspekte zu berücksichtigen (Adressräume, Arten von Synchronisationsmechanismen (Guards, Wächter, Semaphore), Prozesse und Threads, Parallelität im Betriebssystem, Parallelität in virtuellen Maschinen und andere).
%\end{quote}

%\section{Persistenz}

%\begin{quote}
%	Persistenz (Dauerhaftigkeit, Beständigkeit) bedeutet, Daten aus dem (flüchtigen) Hauptspeicher auf ein beständiges Medium (und wieder zurück) zu bringen.
%	Einige der Daten, die ein Software-System bearbeitet, müssen dauerhaft auf einem Speichermedium gespeichert oder von solchen Medien gelesen werden:
%	\begin{itemize}
%		\item Flüchtige Speichermedien (Hauptspeicher oder Cache) sind technisch nicht für dauerhafte Speicherung ausgelegt. Daten gehen verloren, wenn die entsprechende Hardware ausgeschaltet oder heruntergefahren wird.
%		\item Die Menge der von kommerziellen Software-Systemen bearbeiteten Daten übersteigt üblicherweise die Kapazität des Hauptspeichers.
%		\item Auf Festplatten, optischen Speichermedien oder Bändern sind oftmals große Mengen von Unternehmensdaten vorhanden, die eine beträchtliche Investition darstellen.
%	\end{itemize}
%	Persistenz ist ein technisch bedingtes Thema und trägt nichts zur eigentlichen Fachlichkeit eines Systems bei. Dennoch müssen Sie sich als Architekt mit dem Thema auseinander setzen, denn ein erheblicher Teil aller Software-Systeme benötigt einen effizienten Zugriff auf persistent gespeicherte Daten. Hierzu gehören praktisch sämtliche kommerziellen und viele technischen Systeme. Eingebettete Systeme (embedded systems ) gehorchen jedoch oft anderen Regeln hinsichtlich ihrer Datenverwaltung.
%\end{quote}

%\section{Plausibilisierung und Validierung}
%\begin{quote}
%	Wo und wie plausibilisieren und validieren Sie (Eingabe-)daten, etwa Benutzereingaben?
%\end{quote}

%\section{Sessionbehandlung}
%\begin{quote}
%	Sofern ein Roboterarm sich beim ITS-Board erfolgreich registriert hat, kann dieser, wenn erreichbar, angesteuert werden. Dafür wird eine TCP-Verbindung aufgebaut und danach mit RPC kommuniziert. Ein Watchdog auf beiden Seiten überwacht die Verbindung.
%\end{quote}

%\section{Sicherheit}

%\begin{quote}
%	Die Sicherheit von IT-Systemen befasst sich mit Mechanismen zur Gewährleistung von Datensicherheit und Datenschutz sowie Verhinderung von Datenmissbrauch.
%	Typische Fragestellungen sind:
%	\begin{itemize}
%		\item Wie können Daten auf dem Transport (beispielsweise über offene Netze wie das Internet) vor Missbrauch geschützt werden?
%		\item Wie können Kommunikationspartner sich gegenseitig vertrauen?
%		\item Wie können sich Kommunikationspartner eindeutig erkennen und vor falschen Kommunikationspartner schützen?
%		\item Wie können Kommunikationspartner die Herkunft von Daten für sich beanspruchen (oder die Echtheit von Daten bestätigen)?
%	\end{itemize}
%	Das Thema IT-Sicherheit hat häufig Berührung zu juristischen Aspekten, teilweise sogar zu internationalem Recht.
%\end{quote}

%\section{Skalierung}

%\begin{quote}
%Dies Skalierung ist durch das /24 Netz begrenzt. Ansonsten können sich die Roboterarme beliebig mit ihren Kenndaten bei dem ITS-Board anmelden.
%\end{quote}

%\section{Transaktionsbehandlung}

%\begin{quote}
%	Transaktionen sind Arbeitsschritte oder Abläufe, die entweder alle gemeinsam oder garnicht durchgeführt werden. Der Begriff stammt aus den Datenbanken - wichtiges Stichwort hier sind ACID-Transaktionen (atomar, consistent, isolated, durable). Im Bereich von NoSQL-Datenbanken gelten andere Kriterien.
%\end{quote}

%\section{Verteilung}

%\begin{quote}
	
%	Verteilung: Entwurf von Software-Systemen, deren Bestandteile auf unterschiedlichen und eventuell physikalisch getrennten Rechnersystemen ablaufen.
%	Zur Verteilung gehören Dinge wie der Aufruf entfernter Methoden (remote procedure call, RPC), die Übertragung von Daten oder Dokumenten an verteilte Kommunikationspartner, die Wahl passender Interaktionsstile oder Nachrichtenaustauschmuster (etwa: synchron / asynchron, publish- subsribe, peer-to- peer).
%\end{quote}