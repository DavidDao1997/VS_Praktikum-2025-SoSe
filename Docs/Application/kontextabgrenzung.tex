\chapter{Kontextabgrenzung}
Ziel dieses Kapitels ist es, das zu entwickelnde System innerhalb seines fachlichen und technischen Umfelds klar einzugrenzen. Dazu wird das System in Bezug auf seine Aufgaben (fachlicher Kontext), seine Einbettung in die bestehende technische Infrastruktur (technischer Kontext) sowie die definierten externen Schnittstellen beschrieben.

\section{Fachlicher Kontext}

% Das verteilte System soll es ermöglichen beliebig viele (1 - 254) Roboterarme in einem Raum zu steuern. Die Kommunikation zum Nutzer und die Steuerung wird durch ein ITS-Board realisiert. 
% Es wird eine UI angeboten auf dem Rechner, der für die Orientierung für den Nutzer bereitgestellt wird.

Die Applikation ermöglicht es, beliebig viele Roboterarme (zwischen 1 und 254) in einem Raum zu steuern. Die Kommunikation mit dem Nutzer sowie die Steuerung der Roboterarme erfolgen über ein ITS-Board. 
Zur Unterstützung des Nutzers wird auf einer Benutzeroberfläche (UI) bereitgestellt, die eine intuitive Orientierung und Bedienung der Roboterarme ermöglicht.

\subsection{Use Cases}
\begin{tabular}{|p{1.5cm}|p{4cm}|p{8.5cm}|}
	\hline
	\textbf{ID} & \textbf{Name} & \textbf{Beschreibung} \\
	\hline
	U1 & Roboterarm auswählen & Der Nutzer ist in der Lage einen vollständigen Roboterarm auswählen zu können. Der Nutzer hat jederzeit eine Übersicht, welche Roboterarm ausgewählt werden können und ob keiner auswählbar ist. \\
	\hline
	U2 & Bewegung auslösen & Der Nutzer kann die Bewegung des Roboterarms auslösen und bekommt Feedback. \\
	\hline
	U3 & Bewegungsrichtung unterscheiden & Die vom Nutzer ausgelöste Bewegungsrichtung wird dem richtigen Motor zugewiesen. \\
	\hline
	U4 & Neuen Motor erkennen & Die einzelnen Motoren melden sich an. Sobald ein neuer vollständiger Roboterarm (4 Motoren) angesteuert werden kann, ist dieser auswählbar und steuerbar. \\
	\hline
	
	
		
\end{tabular}

\subsection{Fachliche Randbedingungen}

\begin{itemize}
	\item Es können bis zu 254 Roboterarme ausgewählt werden
	\item Jeder Roboterarm hat 4 Motoren, die unabhängig voneinander sind
	\item Teile der Applikation sind auf dem ITS-Board zu implementieren
	\item Die Applikation ist sicher. Personen dürfen nicht zu schaden kommen.
	
\end{itemize}



\section{Technischer Kontext}

% Die Realisierung wird in einem /24 Netzwerk durchgeführt. In diesem wird das ITS-Board liegen. Ebenfalls sind in den Netzwerk die Roboterarme mit den vorgeschalteten Raspberry Pi, der jeweils einen Roboterarm ansteuert. Weitere Hilfmittel wie eine ICC-Cloud können ebenfalls genutzt werden.\\\\
Die Applikation besteht aus einem ITS-Board und jeder Roboterarm wird von einem vorgeschalteten Raspberry Pi angesteuert. Es existiert eine Benutzeroberfläche (UI). Optional kann die Applikation um externe Dienste, die auf anderen Plattformen, wie der ICC-Cloud erweitert werden.

\subsection{Technische Anforderungen an die Middleware}

Um die fachlichen Use Cases umzusetzen, muss die Middleware folgende technische Funktionen bereitstellen:

\begin{itemize}
	\item \textbf{Roboterarm auswählen U1:}\\
	\begin{itemize}
		\item Der Nutzer sucht durch eine Eingabe den gewünschten vollständigen Roboterarm aus
		\item Die Applikation wertet die Auswahl aus, ob die Gruppe auswählbar ist
		\item Die Gruppe aus Motoren wird als ausgewählt für die Ansteuerung markiert
		\item Die Auswahl wird als erfolgreich markiert
		\item dem Nutzer werden die Auswahländerungen angezeigt und bekommt die erfolgreich Mitteilung
	\end{itemize}
	
	\item \textbf{Bewegung auslösen U2:}\\
	\begin{itemize}
		\item Der Nutzer betätigt eine Auswahl für die Bewegungsrichtung
		\item Der Steuerungsbefehl wird inklusive Richtung der Applikation übergeben
		\item Der Steuerungsbefehl wird übersetzt
		\item Die Bewegung wird anhand einer Prozentualen Bewegung durchgeführt
		\item Der erfolgreiche Befehl wird dem Nutzer mitgeteilt
	\end{itemize}
	
	\item \textbf{Bewegungsrichtung unterscheiden U3:}\\
	\begin{itemize}
		\item Der Steuerungsbefehl wird inklusive Richtung der Applikation übergeben
		\item Anhand der Steuerungsrichtung wird entschieden welcher Motor angesprochen wird
		\item Es wird überprüft, ob der Motor angesprochen werden kann/auswählbar ist
		\item Der Steuerungsbefehl wird an den ausgewählten Roboterarm weitergegeben
	\end{itemize}
	
	\item \textbf{Neuen Motor erkennen U4:}\\
	\begin{itemize}
		\item Ein neuer Motor wird angeschlossen
		\item Die Software des Motors sendet einen Anmelde-Anfrage mit dessen Zugehörigkeit und Bewegungsfunktion
		\item Der neue Motor wird eingetragen
		\item Es wird geprüft, ob durch den Motor eine kompletter Roboterarm nun auswählbar ist
		\item dem Nutzer wird ggf. mitgeteilt, dass ein neuer Roboterarm auswählbar ist
	\end{itemize}
	
\end{itemize}


\section{Externe Schnittstellen}

\begin{itemize}
	\item{System:} Der Benutzer kann über ein Bildschirm einzelne erreichbare Roboterarme erkennen. Die Auswahl und Steuern der einzelnen Roboterarme wird durch IO-Buttons realisiert.
	
	% \item{ITS-Board:} Das ITS-Board kommuniziert über einen TCP-Server mit einer eigenen IP-Adresse mit dem System.
   	\item{ITS-Board:} Das ITS-Board kommuniziert über einen leichtgewichtigen UDP-Server mit einer eigenen IP-Adresse mit dem System.

	% \item{Roboterarm:} Der Roboterarm wird mit einer IP-Adresse angesprochen. Dafür ist der vorgeschaltete Raspberry Pi mit einem eigenem UDP-Server zuständig. Dieser stellt ebenfalls eine API für die Steuerung zur Verfügung.  
	\item{Roboterarm:} Der Roboterarm wird über eine IP-Adresse angesprochen. Die Kommunikation erfolgt über einen vorgeschalteten Raspberry Pi, der einen eigenen UDP-Server und eine API zur Steuerung bereitstellt.

	% \item{ICC Cloud:} Die ICC Cloud kommuniziert über eine IP-Adresse. % TODO wording, ist das korrekt formuliert?
	%\item{ICC Cloud:} Die ICC Cloud ist über eine IP-Adresse im Netzwerk erreichbar. %Eventuell so besser? ist es mit UDP/TCP oder Websocket erreichbar?

	
\end{itemize}