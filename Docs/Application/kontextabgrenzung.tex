\chapter{Kontextabgrenzung}
Ziel dieses Kapitels ist es, das zu entwickelnde System innerhalb seines fachlichen und technischen Umfelds klar einzugrenzen. Dazu wird das System in Bezug auf seine Aufgaben (fachlicher Kontext), seine Einbettung in die bestehende technische Infrastruktur (technischer Kontext) sowie die definierten externen Schnittstellen beschrieben.

\section{Fachlicher Kontext}

% Das verteilte System soll es ermöglichen beliebig viele (1 - 254) Roboterarme in einem Raum zu steuern. Die Kommunikation zum Nutzer und die Steuerung wird durch ein ITS-Board realisiert. 
% Es wird eine UI angeboten auf dem Rechner, der für die Orientierung für den Nutzer bereitgestellt wird.

Das verteilte System ermöglicht es, beliebig viele Roboterarme (zwischen 1 und 254) in einem Raum zu steuern. Die Kommunikation mit dem Nutzer sowie die Steuerung der Roboterarme erfolgen über ein ITS-Board. 
Zur Unterstützung des Nutzers wird auf einem Rechner eine Benutzeroberfläche (UI) bereitgestellt, die eine intuitive Orientierung und Bedienung der Roboterarme ermöglicht.

\section{Technischer Kontext}

% Die Realisierung wird in einem /24 Netzwerk durchgeführt. In diesem wird das ITS-Board liegen. Ebenfalls sind in den Netzwerk die Roboterarme mit den vorgeschalteten Raspberry Pi, der jeweils einen Roboterarm ansteuert. Weitere Hilfmittel wie eine ICC-Cloud können ebenfalls genutzt werden.\\\\
Die Realisierung des Systems erfolgt innerhalb eines /24-Netzwerks. In diesem Netzwerk befinden sich das ITS-Board, sowie mehrere Roboterarme.
Jeder Roboterarm wird von einem vorgeschalteten Raspberry Pi angesteuert, der einen eigenen UDP-Server zur Kommunikation bereitstellt. 
Die Benutzeroberfläche (UI) ist über das gleiche Netzwerk erreichbar und kommuniziert per WebSocket mit dem ITS-Board. %Muss ich eventuell ergänzen
Optional kann das System um externe Dienste wie die ICC-Cloud erweitert werden kann.


\section{Externe Schnittstellen}

\begin{itemize}
	\item{System:} Der Benutzer kann über ein Bildschirm einzelne erreichbare Roboterarme erkennen. Die Auswahl und Steuern der einzelnen Roboterarme wird durch IO-Buttons realisiert.
	
	% \item{ITS-Board:} Das ITS-Board kommuniziert über einen TCP-Server mit einer eigenen IP-Adresse mit dem System.
   	\item{ITS-Board:} Das ITS-Board kommuniziert über einen leichtgewichtigen UDP-Server mit einer eigenen IP-Adresse mit dem System.

	% \item{Roboterarm:} Der Roboterarm wird mit einer IP-Adresse angesprochen. Dafür ist der vorgeschaltete Raspberry Pi mit einem eigenem UDP-Server zuständig. Dieser stellt ebenfalls eine API für die Steuerung zur Verfügung.  
	\item{Roboterarm:} Der Roboterarm wird über eine IP-Adresse angesprochen. Die Kommunikation erfolgt über einen vorgeschalteten Raspberry Pi, der einen eigenen UDP-Server und eine API zur Steuerung bereitstellt.

	% \item{ICC Cloud:} Die ICC Cloud kommuniziert über eine IP-Adresse. % TODO wording, ist das korrekt formuliert?
	\item{ICC Cloud:} Die ICC Cloud ist über eine IP-Adresse im Netzwerk erreichbar. %Eventuell so besser? ist es mit UDP/TCP oder Websocket erreichbar?

	
\end{itemize}