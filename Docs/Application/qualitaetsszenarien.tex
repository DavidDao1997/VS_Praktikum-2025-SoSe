\chapter{Qualitätsszenarien}
- Bezug auf section 1.2 
- weniger wichtige Requirements müssen genannt werden

\section{Quality Requirements Overview}



\subsection{Ziele für Software Engineering}
\begin{table}[h!]
	\centering
	\begin{tabular}{p{4cm}|p{5cm}|p{5cm}|}
		\hline
		\textbf{Ziel} & \textbf{Beschreibung} & \textbf{Metrik} \\
		\hline
		Funktionalität &  
		%Der Roboter muss Steuerbefehle korrekt umsetzen, auf Umgebungsdaten reagieren und vordefinierte Aufgaben zuverlässig erfüllen.
		& Alle Abnahmetests werden erfolgreich bestanden
		\\
		\hline
		Zuverlässigkeit & 
		Teilausfälle dürfen den Gesamtsystembetrieb nicht gefährden. Fehlererkennung und -toleranz müssen integriert sein.
		& Das System ist über dem gesamten Abnahmezeitraum stabil (ca. 1,5 h)
		\\
		\hline
		Skalierbarkeit & 
		Zusätzliche Roboterarme oder Komponenten sollen ohne Änderungen an der bestehenden Architektur integrierbar sein. 
		& Es können beliebig viele Roboterarme hinzugefügt und entfernt werden (0 - N)
		\\
		\hline
		Leistung & 
		Reaktionszeiten auf Steuerbefehle und Ereignisse müssen innerhalb definierter Zeitgrenzen liegen. 
		& Reaktionszeit max. 250 ms
		\\
		\hline
		Sicherheit (Safety) & 
		Es bewegt sich immer genau ein Roboterarm. Sollte das System nicht wie gewünscht reagieren, wird ein sicherer Zustand erreicht
		Das System darf unter keinen Umständen eine Gefährdung für Personen/Gegenstände darstellen. Bei Fehlern muss sofort ein sicherer Zustand erreicht werden (z.B. Notstopp). 
		& Reaktionszeit max. 250 ms, bis Roboterarm stoppt
		\\
		\hline
		Wartbarkeit & 
		Der Code muss modular, gut dokumentiert und testbar sein. Fehlerdiagnose und Protokollierung sollen integriert sein. 
		& 
		\\
		\hline
		Portabilität & Die Software soll ohne großen Aufwand auf vergleichbaren Embedded-Systemen lauffähig sein. 
		& 
		\\
		\hline
		Benutzerfreundlichkeit & Konfiguration und Überwachung müssen intuitiv bedienbar und gut visualisiert sein. 
		& Keine Einweisung erforderlich
		\\
		\hline
		Anpassbarkeit & 
		Neue Funktionen, Sensoren oder Regeln sollen ohne tiefgreifende Änderungen am System integrierbar sein. 
		& 
		\\
		\hline
		Kompatibilität & 
		Das System soll mit bestehenden Standards und Protokollen kommunizieren können. 
		& 
		\\
		\hline
	\end{tabular}
	\caption{Qualitätsziele der Software Engineering}
	\label{tab:seziele}
\end{table}

\clearpage
\subsection{Ziele der Verteilte Systeme}
\begin{table}[h!]
	\centering
	\begin{tabular}{p{4cm}|p{5cm}|p{5cm}|}
		\hline
		\textbf{Ziel} & \textbf{Beschreibung} & \textbf{Metrik} \\
		\hline
		Ressourcenteilung  & ...& \\
		Offenheit & ...& \\
		Skalierbarkeit & ...& siehe Tabelle \ref{tab:skalierbarkeit} \\
		Verteilung Transparenz & ...& siehe Tabelle \ref{tab:transparenzen} \\
		\hline
	\end{tabular}
	\caption{Qualitätsziele der Verteilten Systeme}
	\label{tab:vsziele}
\end{table}

\subsubsection{Skalierbarkeit}
\begin{table}[h!]
	\centering
	\begin{tabular}{p{4cm}|p{5cm}|p{5cm}|}
		\hline
		\textbf{Ziel} & \textbf{Metrik} & \textbf{Metrik} \\
		\hline
		Vertikale Skalierung   & ... &\\
		\hline
		Horizontale Skalierung & ...& \\
		\hline
		Räumliche Skalierbarkeit &  & 1 \\
		\hline
		Funktionale Skalierbarkeit & ... & \\
		\hline
		Administrative-Skalierbarkeit & &1 \\
		\hline
	\end{tabular}
	\caption{Skalierbarkeit von verteilten Systemen}
	\label{tab:skalierbarkeit}
\end{table}

\newpage
\subsubsection{Verteilungs-Transparenzen}
\begin{table}[h!]
	\centering
	\begin{tabular}{p{4cm}|p{5cm}|p{5cm}|}
		\hline
		\textbf{Ziel} & \textbf{Beschreibung} & \textbf{Metrik} \\
		\hline
		Zugriffstransparenz   & ...&\\
		\hline
		Lokalitäts-Transparenz  & ...&\\
		\hline
		Migrationstransparenz & ...&\\
		\hline
		Replikationstransparenz &...&\\
		\hline
		Fehlertransparenz &... &\\
		\hline
		Ortstransparenz & .. &\\
		\hline
		Skalierbarkeits-Transparenz & ... & \\
		%Concurrency
		%Relocation
		\hline
	\end{tabular}
	\caption{Verteilungs-Transparenzen}
	\label{tab:transparenzen}
\end{table}

\newpage

\section{Bewertungsszenarien}
\begin{table}[h!]
\centering
\begin{tabular}{p{2cm}|p{5cm}|p{4cm}|p{5cm}}
\hline
\textbf{ID} & \textbf{Context / Background} & \textbf{Source / Stimulus} & \textbf{Metric / Acceptance Criteria} \\
\hline
QS-1 &
Gesamtsystem betriebsbereit. Der Roboter befindet sich im Ruhezustand (Stop). &
Bediener sendet Bewegungsbefehl und das Stromkabel des ITS Board wird gezogen & 
Roboterarm geht innerhalb von 250 ms in den Ruhezustand. \\

\hline
QS-2 &
Gesamtsystem betriebsbereit. Der Roboterarm befindet sich im Ruhezustand. &
Bediener sendet Bewegungsbefehl und das Stromkabel des Raspberry Pi wird herausgezogen. &
Roboterarm innerhalb von 250 ms in den Ruhezustand. 

\end{tabular}
\caption{Bewertungsszenarien nach q42-Modell}
\label{tab:bewertungsszenarien}
\end{table}



\subsection{Abnahmetests}

Die folgenden Abnahmetests dienen dem Nachweis, dass die in Abschnitt \ref{tab:seziele} definierten Qualitätsziele erreicht werden. Jeder Test prüft eine oder mehrere Anforderungen an Funktionalität, Zuverlässigkeit, Leistung oder Sicherheit.

\begin{table}[h!]
\centering
\begin{tabular}{p{1.6cm}|p{4cm}|p{5.5cm}|p{5.5cm}}
\hline
\textbf{Test-ID} & \textbf{Testname} & \textbf{Beschreibung} & \textbf{Erwartetes Ergebnis / Erfolgskriterium} \\
\hline
AT-01 & Heartbeat bei Verbindungsabbruch & Simuliere Verbindungsabbruch (z.B. Ethernet Kabel trennen). Der Heartbeat-Mechanismus muss korrekt auslösen. & Heartbeat fällt aus, System erkennt Verbindungsverlust.\\
\hline
AT-02 & Horizontale Transport Einzelroboter, Hindernis & Ein Roboterarm bewegt ein Objekt horizontal von links nach rechts, hinter einem Hindernis entlang (nicht drüber). & Objekt erreicht korrektes Ziel, keine Kollision mit Hindernis, saubere Bewegung. \\
\hline
AT-03 & Vertikale Transport  Höhenunterschied (20cm) & Objekt wird von unten nach oben über 20cm transportiert und sicher abgelegt. & Objekt liegt stabil, kein Herunterfallen, exakte Position. \\
\hline
AT-04 & Switch zwischen Roboterarmen & Zwischen zwei Robotern wird gewechselt, ohne dass der inaktive Roboter sich bewegt. Keine Teardown-Phase. & Der neue Roboter akzeptiert RPC-Aufrufe, der vorherige ist still. Kein ungewolltes Bewegen. \\
\hline
AT-05 & Horizontale Transport mit zwei Robotern + Switch & Zwei Roboter führen gemeinsam eine horizontale Bewegung aus (wie AT-02), aber mit einem Mid-Weg-Switch von Roboter A auf B. & Objekt bleibt stabil, Bewegung ohne Unterbrechung, reibungsloser Übergang, keine doppelte Bewegung. \\
\hline
AT-06 & Heartbeat-Abbruch beim Roboterwechsel & Beim Wechsel von Roboter A auf B wird gezielt der Heartbeat von A abgebrochen, um Ausfallverhalten zu prüfen. & Heartbeat-Abbruch wird erkannt, System wechselt sicher auf B, kein Deadlock oder Fehlzustand. \\
\hline
AT-07 & Transport unter Netzwerklast & Das System wird während der Bewegung mit zusätzlichem Netzwerkverkehr belastet (z.B. durch künstliche UDP-Flut oder parallele Netzwerkstreams). & Bewegung wird trotz hoher Netzwerklast korrekt ausgeführt, keine Verzögerung, kein Paketverlust mit Auswirkung auf die Bewegung. \\
\hline
\end{tabular}
\caption{Abnahmetests zur Absicherung der Qualitätsanforderungen}
\label{tab:abnahmetests}
\end{table}
