\chapter{Laufzeitsicht}

Betrachtet das dynamische Verhalten des Systems waehrend der Ausfuehrung/Laufzeit.
Wie die jeweiligen Bausteinen sich untereinander verhalten und in beziehung stehen. 

\section{Einleitung - Was?}

Wie fuehren die jeweiligen Bausteine Ihre Funktion aus?\\
Wie interagieren Komponente miteinander\\
 --> Schnittstellen? 
Wie ist der fliessende Datenverkehr der jeweiligen nachrichte\\
Sequenz Diagramm zur visualisierung (bsp. Waehlen des Roboterarms)\\
 --> Wie sieht es aus, wenn ein Fehler geschieht\\
 --> Wie sieht der Start des Systems aus\\

 \textbf{Tipp:}:
 Ein Szenario waehlen, das vieles abdeckt!\\

 \textbf{Motivation?}
 Sehr gut für Stakeholder gedacht - anschaulicher\\
 Hilft zur Beurteilung das System\\
 Hilft es Use-Case zu visualisieren/Realisierung\\
 Hilft die Performance zu verbessern\\
 Troubleshooter Problems (Analysieren und beheben)\\
 Ueberwachung und Wartung des Systems\\
 


\textbf{Form}

\item activity diagram
\item flow charts
\item Sequenz diagram


Sequenzdiagramme zu verschiedenen Szenarien um abläufe zu verdeutlichen (Funktional)




