\chapter{Qualitätsszenarien}
- Bezug auf section 1.2 
- weniger wichtige Requirements müssen genannt werden

\section{Quality Requirements Overview}
- 


\subsection{Ziele für Software Engineering}
\begin{table}[h!]
	\centering
	\begin{tabular}{p{4cm}|p{5cm}|p{5cm}|}
		\hline
		\textbf{Ziel} & \textbf{Beschreibung} & \textbf{Metrik} \\
		\hline
		Funktionalität &  
		%Der Roboter muss Steuerbefehle korrekt umsetzen, auf Umgebungsdaten reagieren und vordefinierte Aufgaben zuverlässig erfüllen.
		& Alle Abnahmetests werden erfolgreich bestanden
		\\
		\hline
		Zuverlässigkeit & 
		Teilausfälle dürfen den Gesamtsystembetrieb nicht gefährden. Fehlererkennung und -toleranz müssen integriert sein.
		& Das System ist über dem gesamten Abnahmezeitraum stabil (ca. 1,5 h)
		\\
		\hline
		Skalierbarkeit & 
		Zusätzliche Roboter oder Komponenten sollen ohne Änderungen an der bestehenden Architektur integrierbar sein. 
		& Es können beliebig viele Roboter hinzugefügt und entfernt werden (0 - N)
		\\
		\hline
		Leistung & 
		Reaktionszeiten auf Steuerbefehle und Ereignisse müssen innerhalb definierter Zeitgrenzen liegen. 
		& Reaktionszeit max. 250 ms
		\\
		\hline
		Sicherheit (Safety) & 
		Es bewegt sich immer genau ein Roboterarm. Sollte das System nicht wie gewünscht reagieren, wird ein sicherer Zustand erreicht
		Das System darf unter keinen Umständen eine Gefährdung für Personen/Gegenstände darstellen. Bei Fehlern muss sofort ein sicherer Zustand erreicht werden (z.B. Notstopp). 
		& Reaktionszeit max. 250 ms, bis Roboterarm stoppt
		\\
		\hline
		Wartbarkeit & 
		Der Code muss modular, gut dokumentiert und testbar sein. Fehlerdiagnose und Protokollierung sollen integriert sein. 
		& 
		\\
		\hline
		Portabilität & Die Software soll ohne großen Aufwand auf vergleichbaren Embedded-Systemen lauffähig sein. 
		& 
		\\
		\hline
		Benutzerfreundlichkeit & Konfiguration und Überwachung müssen intuitiv bedienbar und gut visualisiert sein. 
		& Keine Einweisung erforderlich
		\\
		\hline
		Anpassbarkeit & 
		Neue Funktionen, Sensoren oder Regeln sollen ohne tiefgreifende Änderungen am System integrierbar sein. 
		& 
		\\
		\hline
		Kompatibilität & 
		Das System soll mit bestehenden Standards und Protokollen kommunizieren können. 
		& 
		\\
		\hline
	\end{tabular}
	\caption{Qualitätsziele der Software Engineering}
	\label{tab:seziele}
\end{table}

\clearpage
\subsection{Ziele der Verteilte Systeme}
\begin{table}[h!]
	\centering
	\begin{tabular}{p{4cm}|p{5cm}|p{5cm}|}
		\hline
		\textbf{Ziel} & \textbf{Beschreibung} & \textbf{Metrik} \\
		\hline
		Ressourcenteilung  & ...& \\
		Offenheit & ...& \\
		Skalierbarkeit & ...& siehe Tabelle \ref{tab:skalierbarkeit} \\
		Verteilung Transparenz & ...& siehe Tabelle \ref{tab:transparenzen} \\
		\hline
	\end{tabular}
	\caption{Qualitätsziele der Verteilten Systeme}
	\label{tab:vsziele}
\end{table}

\subsubsection{Skalierbarkeit}
\begin{table}[h!]
	\centering
	\begin{tabular}{p{4cm}|p{5cm}|p{5cm}|}
		\hline
		\textbf{Ziel} & \textbf{Metrik} & \textbf{Metrik} \\
		\hline
		Vertikale Skalierung   & ... &\\
		\hline
		Horizontale Skalierung & ...& \\
		\hline
		Räumliche Skalierbarkeit &  & 1 \\
		\hline
		Funktionale Skalierbarkeit & ... & \\
		\hline
		Administrative-Skalierbarkeit & &1 \\
		\hline
	\end{tabular}
	\caption{Skalierbarkeit von verteilten Systemen}
	\label{tab:skalierbarkeit}
\end{table}

\newpage
\subsubsection{Verteilungs-Transparenzen}
\begin{table}[h!]
	\centering
	\begin{tabular}{p{4cm}|p{5cm}|p{5cm}|}
		\hline
		\textbf{Ziel} & \textbf{Beschreibung} & \textbf{Metrik} \\
		\hline
		Zugriffstransparenz   & ...&\\
		\hline
		Lokalitäts-Transparenz  & ...&\\
		\hline
		Migrationstransparenz & ...&\\
		\hline
		Replikationstransparenz &...&\\
		\hline
		Fehlertransparenz &... &\\
		\hline
		Ortstransparenz & .. &\\
		\hline
		Skalierbarkeits-Transparenz & ... & \\
		%Concurrency
		%Relocation
		\hline
	\end{tabular}
	\caption{Verteilungs-Transparenzen}
	\label{tab:transparenzen}
\end{table}

\newpage

\section{Bewertungsszenarien}
\begin{table}[h!]
\centering
\begin{tabular}{p{2cm}|p{5cm}|p{4cm}|p{5cm}}
\hline
\textbf{ID} & \textbf{Context / Background} & \textbf{Source / Stimulus} & \textbf{Metric / Acceptance Criteria} \\
\hline
QS-1 &
Gesamtsystem betriebsbereit. Der Roboter befindet sich im Ruhezustand (Stop). &
Bediener sendet Bewegungsbefehl und das Stromkabel des ITS Board wird gezogen & 
Roboter geht innerhalb von 250 ms in den Ruhezustand. &

\hline
QS-2 &
Gesamtsystem betriebsbereit. Der Roboter befindet sich im Ruhezustand. &
Bediener sendet Bewegungsbefehl und das Stromkabel des Raspberry Pi wird herausgezogen. &
Roboter innerhalb von 250 ms in den Ruhezustand. 

\end{tabular}
\caption{Bewertungsszenarien nach q42-Modell}
\label{tab:bewertungsszenarien}
\end{table}
