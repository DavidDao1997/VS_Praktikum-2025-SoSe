\chapter{Kontextabgrenzung}
Ziel dieses Kapitels ist es, das zu entwickelnde System innerhalb seines fachlichen und technischen Umfelds klar einzugrenzen. Dazu wird das System in Bezug auf seine Aufgaben (fachlicher Kontext), seine Einbettung in die bestehende technische Infrastruktur (technischer Kontext) sowie die definierten externen Schnittstellen beschrieben.

\section{Fachlicher Kontext}

Das verteilte System soll es ermöglichen beliebig viele (1 - 254) Roboterarme in einem Raum zu steuern. Die Kommunikation zum Nutzer und die Steuerung wird durch ein ITS-Board realisiert. 

\section{Technischer Kontext}

Die Realisierung wird in einem /24 Netzwerk durchgeführt. In diesem wird das ITS-Board liegen. Ebenfalls sind in den Netzwerk die Roboterarme mit den vorgeschalteten Raspberry Pi, der jeweils einen Roboterarm ansteuert. Weitere Hilfmittel wie eine ICC-Cloud können ebenfalls genutzt werden.

\section{Externe Schnittstellen}

\begin{itemize}
	\item{System:} Der Benutzer kann über ein Bildschirm einzelne erreichbare Roboterarme erkennen. Die Auswahl und Steuern der einzelnen Roboterarme wird durch IO-Buttons realisiert.
	
	\item{ITS-Board:} Das ITS-Board kommuniziert über einen TCP-Server mit einer eigenen IP-Adresse mit dem System.
	
	\item{Roboterarm:} Der Roboterarm wird mit einer IP-Adresse angesprochen. Dafür ist der vorgeschaltete Raspberry Pi mit einem eigenem TCP-Server zuständig. Dieser stellt ebenfalls eine API für die Steuerung zur Verfügung.  
	
	\item{ICC Cloud:} Die ICC Cloud kommuniziert über eine IP-Adresse. % TODO wording, ist das korrekt formuliert?
	
	
\end{itemize}