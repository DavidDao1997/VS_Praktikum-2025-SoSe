

\chapter{Randbedingungen}

Dieses Kapitel beschreibt die Rahmenbedingungen, unter denen die Middleware entworfen und implementiert werden muss.

\section{Technische Randbedingungen}

TODO Zeichnung

\begin{itemize}
    \item \textbf{Betriebsumgebung:}  Die Middleware muss auf einer heterogenen Umgebung aus verschiedenen Hardware-Plattformen und Betriebssystemen laufen können. Im spezifischen Projektkontext umfasst dies mindestens ein ITS-Board (STM32F4) und mehrere Raspberry Pis.
    
    \item \textbf{Netzwerk:}  
    \begin{itemize}
        \item Die Kommunikation findet innerhalb eines begrenzten Netzwerks statt, im Projektkontext ein /24 Netzwerk. Die Middleware baut auf den grundlegenden Kommunikationsdiensten des Betriebssystems und Netzwerks auf.
    \end{itemize}
    \item \textbf{Hardware:} Die Middleware muss auf der Hardware laufen, die für das Steuerungssystem verwendet wird: ITS-Board und Raspberry Pis. 
    \item \textbf{Anbindung:} Die Middleware muss in der Lage sein, mit der Anwendungsschicht und den System-/Netzwerkschichten zu interagieren.

    
\end{itemize}

\section{Organisatorische Randbedingungen}
\begin{itemize}
    \item \textbf{Zeit:} Entwicklungszeitraum beträgt 12 Wochen. 
    \item \textbf{Vorwissen:} Einige Konzepte und Herangehensweisen werden erst im Laufe der 12 Wochen gelernt.
    \item \textbf{Budget:} Es steht kein Budget zur Verfügung.

\end{itemize}

%\section{Konventionen}


