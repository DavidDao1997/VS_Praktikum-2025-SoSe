

\chapter{Randbedingungen}

Dieses Kapitel beschreibt die Rahmenbedingungen, unter denen die Middleware entworfen und implementiert werden muss.

\section{Technische Randbedingungen}

%TODO Zeichnung

\begin{itemize}
    \item \textbf{Betriebsumgebung:}  Die Middleware muss auf einer heterogenen Umgebung aus verschiedenen Hardware-Plattformen und Betriebssystemen laufen können. Im spezifischen Projektkontext umfasst dies mindestens ein ITS-Board (STM32F4) und mehrere Raspberry Pis.
    
    \item \textbf{Netzwerk:}  
    \begin{itemize}
        \item Die Kommunikation findet innerhalb eines begrenzten Netzwerks statt, im Projektkontext ein /24 Netzwerk. Die Middleware baut auf den grundlegenden Kommunikationsdiensten des Betriebssystems und Netzwerks auf.
        \item Es wird mit IPv4 kommuniziert.
    \end{itemize}
    \item \textbf{Hardware:} Die Middleware muss auf der Hardware laufen, die für das Steuerungssystem verwendet wird: ITS-Board und Raspberry Pi 3. 
    \begin{itemize}
    	\item ITS-Board (STM32 Nucleo-144 Board): 
    	\begin{itemize}
    		\item OS: Kein Betriebssystem vorhanden (RTOS möglich)
    		\item CPU: STM32F4
    		
    	\end{itemize}
    	\item Raspberry Pi 3
    	\begin{itemize}
    		\item OS: Raspian GNU/Linux 9
    		\item CPU: ARMv7 Processor rev 5
    		\item RAM: 927 MB
    		
    	\end{itemize}
    \end{itemize}
    \item \textbf{Anbindung:} Die Middleware muss in der Lage sein, mit der Anwendungsschicht und den System-/Netzwerkschichten zu interagieren.
    \item \textbf{Sprachen:} Die Middleware muss in den Sprachen C und JAVA umgesetzt werden. 

    
\end{itemize}

\section{Organisatorische Randbedingungen}
\begin{itemize}
    \item \textbf{Zeit:} Entwicklungszeitraum beträgt 12 Wochen. 
    \item \textbf{Vorwissen:} Einige Konzepte und Herangehensweisen werden erst im Laufe der 12 Wochen gelernt.
    \item \textbf{Budget:} Es steht kein Budget zur Verfügung.

\end{itemize}

%\section{Konventionen}


