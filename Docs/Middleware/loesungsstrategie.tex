\chapter{Lösungsstrategie}


\textcolor{red}{TODO: Weitere Funktionen (atomar))}

\begin{longtable}{|>{\raggedright\arraybackslash}p{4cm}|>{\raggedright\arraybackslash}p{5cm}|>{\raggedright\arraybackslash}p{5cm}|}
	\caption{Funktionsbeschreibungen} \label{tab:loesungsstrategie} \\
	\hline
	Funktion & Beschreibung & Vor-Nachbedingungen \\
	\hline
	\endfirsthead
	
	\hline
	Funktion & Beschreibung & Vor-Nachbedingungen \\
	\hline
	\endhead
	
	\hline
	\endfoot
	
	
	void register(String name, String socket) & 
	\textbf{Schnittstelle zur Applikation} für eine Node (Servo), um sich zu beim System zu registrieren. Identifier ist für jeden Servo einzigartig, socket ist die IPv4 Adresse plus Port der Applikation. 
	& Jede Node (Servo) muss seinen eigenen Identifier und Socket vor der Registrierung kennen. 
	\\
	\hline
	int registerNewNode(String name, String socket) & 
	Registriert neuen Empfänger in der Namensauflösung 
	& Name ist noch nicht vorhanden; return 0 oder -1
	\\
	\hline
	void invoke(name, Marshallable call)
	& \textbf{Schnittstelle zur Applikation} um RPCs an entfernte Ziele zu senden. 
	& IDL der Middleware kann den Typen Marshallable der Applikation serialisieren
	\\
	\hline
	void call(Marshallable call)
	& \textbf{Schnittstelle zur Applikation} um RPCs von entfernten Zielen zu empfangen
	& IDL der Middleware kann den Typen Marshallable der Applikation unmarshallen
	\\
	\hline
	int marshal(Marshallable* call, char* buffer, size bufSize)
	& Führt Marshalling durch
	& Marshallable und buffer darf nicht NULL sein. Nach: return 0 oder -1
	\\
	\hline
	int unmarshal(const char* buffer, Marshallable* call)
	& Führt Unmarshalling durch
	& Marshallable und buffer darf nicht NULL sein. Nach: return 0 oder -1
	\\
	\hline
	void updateAvailableNodes(List<Node> nodes)
	& Liste mit allen verfügbaren Nodes, um diese der Applikation mitzuteilen. 
	& Es existiert eine Liste die auch von der Applikation gelesen werden kann (Liste muss angepasst werden an die )
	\\
	\hline
	void forwardCall(String* target, Marshallable* call)
	& Leitet den ankommenden RPC an das Target weiter
	& Das Target muss existieren.
	\\
	\hline
	\textcolor{red}{TODO: CHECK wegen Watchdog} void heartbeat(Identifier ident)	
	& Check ob Node verfügbar	
	& Vor: Node ist registriert; Nach: Timeout-Zähler zurückgesetzt
	\\
	\hline
	void unregister(Identifier ident)
	& Entfernt eine Node aus dem System, zb wenn nicht mehr erreichbar anhand seines Identifiers
	& Vor: Node ist registriert; Nach: Node ist aus Liste entfernt
	\\
	\hline
	 
\end{longtable}


\begin{longtable}{|>{\raggedright\arraybackslash}p{4cm}|>{\raggedright\arraybackslash}p{10cm}|}
	\caption{Lösungstrategien Ziele} \label{tab:loesungsstrategieZiele} \\
	\hline
	Ziel & Strategie \\
	\hline
	\endfirsthead
	
	\hline
	Ziel & Strategiee\\
	\hline
	\endhead
	
	\hline
	\endfoot
	
	Kommunikation &
	Asynchrones Remote Procedure Call (RPC) als Kommunikationsmechanismus. Registrierungsvorgänge werden persistent gespeichert, während Steuerbefehle transient und zeitkritisch behandelt werden.  
	\\
	\hline
	Namensauflösung
	& Flacher Namensraum. Der Name besteht aus eindeutiger Node-ID, und IPv4 Socketinformationen. 
	\\
	\hline
	Fehlerbehandlung
	& Logging, Timeouts und Wiederholungen von Weiterleitungen. Erkannte Fehler werden der Applikation mitgeteilt \textcolor{red}{TODO: Funktion bestimmen}
	\\
	\hline
	Marshalling
	& Verwendung eines einheitlichen Marshallable-Typs und einer IDL zur Beschreibung und Serialisierung komplexer Funktionsaufrufe.
	\\
	\hline
	Transparenz
	& Middleware abstrahiert Verbindungsdetails, Adressierung und Datenformate für die Applikation vollständig (Standort-, Zugriffstransparenz).
	\\
	\hline
	Sicherheit / Safety
	& \textcolor{red}{TODO: Watchdog }
	\\
	\hline
	
\end{longtable}


