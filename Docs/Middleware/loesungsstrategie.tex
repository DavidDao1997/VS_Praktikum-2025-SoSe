\chapter{Lösungsstrategie}


\textcolor{red}{TODO: Weitere Funktionen (atomar))}

\begin{longtable}{|>{\raggedright\arraybackslash}p{4cm}|>{\raggedright\arraybackslash}p{5cm}|>{\raggedright\arraybackslash}p{5cm}|}
	\caption{Funktionsbeschreibungen} \label{tab:loesungsstrategie} \\
	\hline
	Funktion & Beschreibung & Vor-Nachbedingungen \\
	\hline
	\endfirsthead
	
	\hline
	Funktion & Beschreibung & Vor-Nachbedingungen \\
	\hline
	\endhead
	
	\hline
	\endfoot
	
	
	void register(Identifier ident, enum functionality) & 
Schnittstelle für eine Node (Servo), um sich zu beim System zu registrieren. Identifier ist für jeden Servo einzigartig, functionality ist ein enum mit Aufgabenbereich an einem Roboterarm. 
	& Jede Node (Servo) muss seinen eigenen Identifier und Funktionalität vor der Registrierung kennen. 
	\\
	\hline
	Node createNewNode (Identifier ident, enum functionality)
	& Übersetzt externe Node zu internen Node Code
	&
	\\
	\hline
	void updateAvailableNodes(List<Node> nodes)
	& Liste mit allen verfügbaren Nodes, um diese der Applikation mitzuteilen. 
	& 
	\\
	\hline
	void heartbeat(Identifier ident)	
	& Check ob Node verfügbar	
	& Vor: Node ist registriert; Nach: Timeout-Zähler zurückgesetzt
	\\
	\hline
	void unregister(Identifier ident)
	& Entfernt eine Node aus dem System, zb wenn nicht mehr erreichbar anhand seines Identifiers
	& Vor: Node ist registriert; Nach: Node ist aus Liste entfernt
	\\
	\hline
	void choose(NodeGrp nodeGrpIdent) 
	& Schnittstelle für Applikation, um Node Gruppe zu wechseln
	&
	\\
	\hline
	void markActive(NodeGrp nodeGrpIdent)
	& Node Gruppe wird als aktiv makiert
	& Vor: Applikation hat wechsel beauftragt; Nach: Node Gruppe wird als aktiv markiert, alle Steuerbefehle beziehen sich auf diese Gruppe
	\\
	\hline
	void move(Direction dir) 
	& Schnittstelle für Applikation um Node zu bewegen 
	& Vor: Node Gruppe ist ausgewählt und verfügbar; Nach: Bewegung wird ausgeführt
	\\
	\hline
	Node readFromNodeList(Direction dir)
	& Gibt aktiven Node zurück mit passender Funktion
	&
	\\
	\hline
	void isActive(Direction dir) 
	& Schnittstelle für Applikation um Node zu bewegen 
	& Vor: Node Gruppe ist ausgewählt und verfügbar; Nach: Bewegung wird ausgeführt
	\\
	\hline
	void moveNode(Identifier ident)
	& ruft entsprechende Node Schnittstelle auf
	& 
\end{longtable}


\textcolor{red}{TODO: Identifier muss ausgearbeitet werden (Allgemein Service Discovery und Speicherung) sowie Rückgabetyp Node}\\


\begin{longtable}{|>{\raggedright\arraybackslash}p{4cm}|>{\raggedright\arraybackslash}p{10cm}|}
	\caption{Lösungstrategien Ziele} \label{tab:loesungsstrategieZiele} \\
	\hline
	Ziel & Strategie \\
	\hline
	\endfirsthead
	
	\hline
	Ziel & Strategiee\\
	\hline
	\endhead
	
	\hline
	\endfoot
	
	Kommunikation &
	Asynchrones Remote Procedure Call (RPC) als Kommunikationsmechanismus. Registrierungsvorgänge werden persistent gespeichert, während Steuerbefehle transient und zeitkritisch behandelt werden.  
	\\
	\hline
	Service-Discovery &
	Flacher Namensraum mit Gruppierung der Nodes. Der Name besteht aus eindeutiger Node-ID, Funktion und Zugehörigkeit zum Roboterarm-Verbund. 
	\\
	\hline
	Namensauflösung
	& IPv4 Adresse und dazugehöriger Node wird in einen Namen übersetzt, der die Node zu einem Verbund zugehörig macht.  
	\\
	\hline
	Fehlerbehandlung
	& 
	
	
\end{longtable}


