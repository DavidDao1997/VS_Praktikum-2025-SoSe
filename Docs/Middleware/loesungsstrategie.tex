\chapter{Lösungsstrategie}


\begin{longtable}{|>{\raggedright\arraybackslash}p{4cm}|>{\raggedright\arraybackslash}p{5cm}|>{\raggedright\arraybackslash}p{5cm}|}
	\caption{Funktionsbeschreibungen} \label{tab:loesungsstrategie} \\
	\hline
	Funktion & Beschreibung & Vor-Nachbedingungen \\
	\hline
	\endfirsthead
	
	\hline
	Funktion & Beschreibung & Vor-Nachbedingungen \\
	\hline
	\endhead
	
	\hline
	\endfoot
	
	
	void register(Identifier ident, enum functionality) & 
Schnittstelle für eine Node (Servo), um sich zu beim System zu registrieren. Identifier ist für jeden Servo einzigartig, functionality ist ein enum mit Aufgabenbereich an einem Roboterarm. 
	& Jede Node (Servo) muss seinen eigenen Identifier und Funktionalität vor der Registrierung kennen. 
	\\
	\hline
	Node createNewNode (Identifier ident, enum functionality)
	& Translates externe Node zu internen Node Code
	&
	\\
	\hline
	void updateAvailableNodes(List<Node> nodes)
	& Liste mit allen verfügbaren Nodes, um diese der Applikation mitzuteilen
	& 
	\\
	\hline
	bool checkAvailability(Node node)
	& Checks ob Node verfügbar	
	&
	\\
	\hline
	void choose(NodeGrp nodeGrpIdent) 
	& Schnittstelle für Applikation, um Node Gruppe zu wechseln
	&
	\\
	\hline
	void markActive(NodeGrp nodeGrpIdent)
	& Node Gruppe wird als aktiv makiert
	& Vor: Applikation hat wechsel beauftragt; Nach: Node Gruppe wird als aktiv markiert, alle Steuerbefehle beziehen sich auf diese Gruppe
	\\
	\hline
	void move(Direction dir) 
	& Schnittstelle für Applikation um Node zu bewegen 
	& Vor: Node Gruppe ist ausgewählt und verfügbar; Nach: Bewegung wird ausgeführt
	\\
	\hline
\end{longtable}


\begin{longtable}{|>{\raggedright\arraybackslash}p{4cm}|>{\raggedright\arraybackslash}p{5cm}|>{\raggedright\arraybackslash}p{5cm}|}
	\caption{Lösungstrategien Ziele} \label{tab:loesungsstrategieZiele} \\
	\hline
	Ziel & Strategie & Alternative\\
	\hline
	\endfirsthead
	
	\hline
	Ziel & Strategie & Altenative\\
	\hline
	\endhead
	
	\hline
	\endfoot
	
	
	Kommunikation
	& RPC, ..... 
	& 
	\\
	\hline
	Service-Discovery-Strategie
	& flacher Namensraum (aber gruppiert); Namenaufbau: ...; Zu jeder Node wird Funktion und IP-Adresse gespeichert
	& 
	\\
	\hline

	
\end{longtable}


