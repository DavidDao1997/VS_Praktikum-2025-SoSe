\chapter{Einführung und Ziele}

Diese Dokumentation beschreibt Entwurf und Kontext einer Middleware, die im Rahmen des Praktikums \glqq Verteilte Systeme SoSe 2025\grqq{} entwickelt wird. Ziel des Praktikums ist der Entwurf und die Implementierung eines verteilten Steuerungssystems für Roboterarme. Die Middleware bildet dabei eine zentrale Komponente, die die Komplexität der Verteilung für die Anwendungsebene abstrahieren soll.


\section{Aufgabenstellung}
Die Hauptaufgabe der zu entwickelnden Middleware besteht darin, die Kommunikation und Koordination zwischen den verteilten Komponenten des Steuerungssystems zu ermöglichen. Sie soll eine Abstraktionsschicht zwischen der eigentlichen Anwendungslogik (Roboterarmsteuerung) und den darunterliegenden Betriebssystem- und Netzwerkdiensten bilden. Gemäß den Prinzipien verteilter Systeme nach Tanenbaum \& van Steen, die auch im zugrundeliegenden Skript referenziert werden, verfolgt die Middleware das Ziel, die Verteilung weitestgehend zu verbergen (Transparenz).
Zu den Kernfunktionen, die eine Middleware typischerweise bereitstellt und die hier adressiert werden sollen, gehören:

\begin{itemize}
	\item Zuständig für die Verteilungstransparenz	
	\item Bereitstellung der Kommunikation mittels standardisierter Protokolle. Dies wird (asynchrone) Kommunikation-Pattern umfassen, wie z.B. Remote Procedure Calls (RPC).
	\item Unterstützung zur Konvertierung von Datenformaten zwischen Systemen.
	\item Bereitstellung von Protokollen zur Namensauflösung (Service Discovery), um Ressourcen einfach zu identifizieren und zu referenzieren.
	\item Mechanismen der Skalierbarkeit, wie z.B. Replikation
	\item Security Mechanismus, dass nicht akzeptierte Nachrichten verworfen werden. 
\end{itemize}

Die Middleware dient als Vermittler, der zwischen der Applikation und der Runtime/OS Schicht Daten- und Informationsaustausch ermöglicht.


Zusammengefasst soll die Middleware Steuerung der einzelnen Nodes (Servos) eines jeden im System befindlichen Roboterarms ermöglichen. Dazu gehören die Erkennung jeder Node, und die Weiterleitung der Steuerbefehle.

\newpage
\section{Qualitätsziele}
% Aus VS Skript Kapitel 2.4 Seite 23
% TODO zu jeden Punkt eine Definition oder entfernen
% TODO messbar definieren
\begin{longtable}{|>{\raggedright\arraybackslash}p{4cm}|>{\raggedright\arraybackslash}p{5cm}|>{\raggedright\arraybackslash}p{5cm}|}
	\caption{Qualitätsziele der Software Engineering} \label{tab:seziele} \\
	\hline
	Ziel & Beschreibung & Metrik \\
	\hline
	\endfirsthead
	
	\hline
	Ziel & Beschreibung & Metrik \\
	\hline
	\endhead
	
	\hline
	\endfoot
	
	
	Zuverlässigkeit & 
	Fehler dürfen den Betrieb nicht gefährden. Fehlererkennung und -toleranz müssen integriert sein.
	& Das System ist über dem gesamten Abnahmezeitraum stabil (ca. 1,5 h).Definiert aufgetretene Fehler werden kommuniziert.
	\\
	\hline
	Skalierbarkeit & 
	Zusätzliche Nodes oder Komponenten sollen ohne Änderungen an der bestehenden Architektur integrierbar sein. 
	% Die Middleware muss Mechanismen (z.B. Replikation, Caching) bieten, um mit zunehmender Anzahl von Teilnehmern oder der Datenmenge zurechtzukommen.
	& Es können bis zu 253 Nodes hinzugefügt und entfernt werden
	\\
	\hline
	%Leistung & 
	%Reaktionszeiten auf Steuerbefehle und Ereignisse müssen innerhalb definierter Zeitgrenzen liegen. 
	%& max. 200 ms bis Befehlsausführung
	%\\
	%\hline
	Wartbarkeit & 
	Der Code muss übersichtlich sein, gut dokumentiert sein und wenig Komplexität enthalten. 
	& Zyklomatische Komplexität $\leq$ 10 und LOC $\leq$ 30 pro Methode/Funktion exklusive Kommentar  
	
	\\
	\hline
	
	%Portabilität & %Die Software soll ohne großen Aufwand auf vergleichbaren Embedded-Systemen lauffähig sein. 
	%& nicht relevant
	%\\
	%\hline
	%Benutzerfreundlichkeit & Bedienung ist intuitiv, sodass die Nutzer möglichst wenig Zeit mit der Einarbeitung in die Bedienung benötigen. 
%	& Keine Einweisung erforderlich
	%\\
	%\hline
	%Anpassbarkeit & 
	%Neue Funktionen, Sensoren oder Regeln sollen ohne tiefgreifende Änderungen am System integrierbar sein. 
	%& 
	%\\
	%\hline
	%Kompatibilität & 
	%Das System soll mit bestehenden Standards und Protokollen kommunizieren können. 
	%& 
	%\\
	%\hline
	Ressourcenteilung  & Alle dem Netzwerk hinzugefügten Nodes können sich registrieren und anschließend miteinander kommunizieren & Das Namensregister ist im gesamten Netzwerk verfügbar\\
	\hline
	Offenheit & 
	\parbox[t]{5cm}{Zugänglichkeit\\Interoperabilität\\Portabilität} 
	& Einzelne Softwarekomponenten können, ausgetauscht oder portiert werden, um zb Standards bzgl. Kommunikation und Datenstruktur zu tauschen oder um die Plattform zu wechseln. Die Funktionalität bleibt gleich\\
	\hline
	Zugriffstransparenz & Die Umsetzung Kommunikation zwischen den Nodes ist für den Benutzer nicht erkennbar & Die Applikation kommuniziert über Namen. Die eigentliche Kommunikation bleibt versteckt.\\
	\hline
	Lokalitäts-Transparenz  & Die Netzwerk- und Softwarestruktur ist nach außen unsichtbar  & Das Interface nach außen ist eineiheitlich und verschleiert die Implementierung\\
	
	
	
	
\end{longtable}
\clearpage
\section{Stakeholder}
\begin{table}[h!]
	\caption{Interessen der Stakeholder}
	\label{tab:stakeholder}
	\centering
	\begin{tabular}{|p{2.5cm}|p{10cm}|}
		\hline
		\textbf{Stakeholder} & \textbf{Interesse}  \\
		\hline
		Betreiber 
		&\parbox{10cm}{\begin{itemize}
				\item Portabilität: Das System kann auf verschiedenen Plattformen betrieben werden.
				\item Zuverlässigkeit: Die Middleware kann über den gesamten benötigten Zeitraum ohne Ausfälle genutzt werden
		\end{itemize}}
		\\
		\hline
		Entwicklerteam Middleware
		&\parbox{10cm}{ \begin{itemize}
				\item Wartbarkeit
				\item Portabilität: Das System kann auf verschiedenen Plattformen betrieben werden (z.B Testen)
				\item Austauschbarkeit: Softwaremodule können ohne großen Aufwand ersetzt werden
		\end{itemize}}
		\\
		\hline
		Entwicklerteam Applikation
		&\parbox{10cm}{ \begin{itemize}
				\item Middleware bietet vollständige Funktionalität
				\item Middleware-Schnittstellen sind vollständig beschrieben.
				\item Zuverlässigkeit und Reaktionszeit der von der Middleware bereitgestellten Kommunikationsdienste.
		\end{itemize}}
		\\
		\hline
		
		Professor & \parbox{10cm}{\begin{itemize}
				\item Zugang zu allen Arbeitsmitteln zwecks Bewertung und Kontrolle
				\item Das Endprodukt besitzt alle geforderten Funktionalitäten
		\end{itemize}}
		\\
		\hline
	\end{tabular}
\end{table}