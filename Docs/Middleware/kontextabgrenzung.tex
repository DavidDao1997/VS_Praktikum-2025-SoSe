\chapter{Kontextabgrenzung}


\section{Fachlicher Kontext}


\subsection{Akteure und Rollen}

\begin{tabular}{|p{4cm}|p{10cm}|}
	\hline
	\textbf{Akteur / Rolle} & \textbf{Beschreibung} \\
	\hline
	Applikation & Externes Steuerungssystem, das über eine Schnittstelle Befehle an die Middleware sendet, z.\,B. zur Auswahl oder Bewegung eines Roboterarms. \\
	\hline
	Node (z.\,B. Servo) & Ein einzelner Aktor, der physisch mit einem Roboterarm verbunden ist. Meldet sich eigenständig bei der Middleware an und gehört zu einer logischen Gruppe. \\
	\hline
	Node-Gruppe & Eine logische Einheit mehrerer Nodes, die gemeinsam einen vollständigen Roboterarm darstellen. Nur eine Gruppe kann gleichzeitig aktiv sein. \\
	\hline
\end{tabular}

\subsection{Fachliche Aufgaben (Use Cases)}

\begin{tabular}{|p{1.5cm}|p{4cm}|p{8.5cm}|}
	\hline
	\textbf{ID} & \textbf{Name} & \textbf{Beschreibung} \\
	\hline
	U1 & Node registrieren & Eine Node registriert sich bei der Middleware mit einem eindeutigen Identifier und einer Funktionsbeschreibung. Sie wird anschließend einer Node-Gruppe (Roboterarm) zugeordnet. \\
	\hline
	U2 & Node-Gruppe auswählen & Die Applikation wählt eine existierende Node-Gruppe aus, die als Ziel für Steuerungsbefehle verwendet werden soll. \\
	\hline
	U3 & Node-Gruppe bewegen & Die Middleware empfängt Bewegungsbefehle von der Applikation und leitet diese an entsprechende Nodes der aktiven Node-Gruppe weiter. \\
	\hline
	U4 & Verfügbarkeit prüfen & Die Middleware prüft, ob alle Nodes einer gewählten Gruppe erreichbar und einsatzbereit sind, bevor Steuerungsbefehle ausgeführt werden. \\
	\hline
	U5 & Verfügbarkeit propagieren & Die Middleware muss der Applikation in regelmäßigen Abständen mitteilen welche Node-Gruppen fehlerfrei erreichbar sind.\\
	\hline
\end{tabular}

\subsection{Fachliche Randbedingungen}

\begin{itemize}
	\item Es können bis zu 253 Nodes gleichzeitig betrieben werden.
	\item Jede Node gehört genau einer Node-Gruppe an.
	\item Steuerungsbefehle dürfen nur an Gruppen gesendet werden, die als aktiv markiert sind.
	\item Die Middleware muss mit dem Ausfall oder der Nichterreichbarkeit einzelner Nodes robust umgehen können.
	\item Die Systemsicherheit hat höchste Priorität: Bewegungen fehlerhafter Gruppen müssen verhindert werden.
\end{itemize}

\section{Technischer Kontext}

Die Middleware positioniert sich technisch als Vermittlungsschicht zwischen der Anwendungsebene und den darunterliegenden Betriebssystem- und Kommunikationsdiensten. Sie ist über mehrere Maschinen verteilt, die über ein lokales /24 Netzwerk miteinander verbunden sind.
Die Middleware nutzt Netzwerkprotokolle, insbesondere TCP und UDP, sowie Betriebssystemfunktionen, um eine zuverlässige Kommunikation zu gewährleisten. Gleichzeitig bietet sie den Anwendungen eine abstrahierte, einheitliche Schnittstelle, die die Heterogenität der zugrundeliegenden Systeme verbirgt.
Die Middleware stellt darüber hinaus eigene Kommunikationsprotokolle bereit, die speziell auf die Anforderungen der verteilten Steuerung und Koordination der Roboterarme zugeschnitten sind. Hierzu zählen Mechanismen zur Nachrichtenvermittlung, Statusüberwachung und Fehlerbehandlung, um eine hohe Zuverlässigkeit und Fehlertoleranz sicherzustellen.
Zusammenfassend erfüllt die Middleware folgende technische Aufgaben:
\begin{itemize}
	\item Verbergen der Heterogenität der zugrundeliegenden Hardware und Betriebssysteme
	\item Bereitstellung einer einheitlichen Kommunikationsschnittstelle für die darüber liegenden Anwendungen
	\item Sicherstellung der zuverlässigen und performanten Kommunikation zwischen verteilten Komponenten
	\item Implementierung von Mechanismen zur Erkennung und Behandlung von Fehlerzuständen im Netzwerk und auf den Nodes
\end{itemize}

\subsection{Funktionale Anforderungen an die Middleware}

Um die fachlichen Use Cases umzusetzen, muss die Middleware folgende technische Funktionen bereitstellen:

\begin{itemize}
	\item \textbf{Registrierungsmanagement (U1):}  
	Verwaltung und Verarbeitung von Node-Registrierungen mit eindeutigen Identifikatoren und Funktionsbeschreibungen.  
	Persistenz der Node-Informationen und Zuordnung zu Node-Gruppen.
	
	\item \textbf{Gruppenverwaltung (U2):}  
	Auswahl, Wechsel und Markierung aktiver Node-Gruppen zur Steuerung. Verwaltung des Gruppenstatus.
	
	\item \textbf{Steuerungsweiterleitung (U3):}  
	Empfang von Steuerbefehlen von der Applikation und Weiterleitung dieser Befehle an die Nodes der aktiven Gruppe.
	
	\item \textbf{Verfügbarkeitsprüfung (U4):}  
	Überwachung des Status und der Erreichbarkeit aller Nodes innerhalb einer Gruppe, inklusive Fehlererkennung und Meldung an die Applikation.
	\item \textbf{Verfügbare Nodegruppen propagieren (U5):}  
	Mitteilung an Applikation, welche Node-Gruppen verfügbar und ansteuerbar sind. \textcolor{red}{TODO: Entweder Schnittstelle anbieten und Applikation entscheiden lassen, oder Schnittstelle der Applikation nutzen!?}
	
	\item \textbf{Fehlertoleranz und Sicherheit:}  
	Mechanismen zur Behandlung von Kommunikationsausfällen, erneutes Senden von Nachrichten und Schutz vor Fehlfunktionen einzelner Nodes.
\end{itemize}

\section{Externe Schnittstellen}

Die Middleware besitzt zwei Schnittstellen:

\begin{itemize}
	\item \textbf{Schnittstelle zur Anwendungsebene:}  
	Bietet eine einheitliche API, über die Anwendungen (z.\,B. auf ITS-Board oder Raspberry Pis) verteilte Dienste nutzen, Nachrichten senden und empfangen können – unabhängig von Netzwerkdetails oder physikalischer Verteilung.
	
	\item \textbf{Schnittstelle zu System- und Netzwerkschichten:}  
	Nutzt Betriebssystem- und Netzwerkdienste (TCP/IP, IPv4) zur Nachrichtenübermittlung und Ressourcenverwaltung. Diese Schnittstelle ist für die Anwendungen verborgen.
\end{itemize}